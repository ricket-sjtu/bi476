\documentclass[table,10pt]{beamer}

\mode<presentation>{
%\usetheme{Goettingen}
\usetheme{Boadilla}
\usecolortheme{default}
}
\usepackage{CJK}
\usepackage{graphicx}
\usepackage{amsmath, amsopn}
\usepackage{xcolor}
\usepackage[english]{babel}
\usepackage[T1]{fontenc}
\usepackage[latin1]{inputenc}
\usepackage{enumerate}
\usepackage{multirow}
\usepackage{url}
\ifx\hypersetup\undefined
	\AtbBeginDocument{%
		\hypersetup{unicode=true,pdfusetitle,
bookmarks=true,bookmarksnumbered=false,bookmarksopen=false,
breaklinks=false,pdfborder={0 0 0},pdfborderstyle={},backref=false,colorlinks=false}
	}
\else
	\hypersetup{unicode=true,pdfusetitle,
bookmarks=true,bookmarksnumbered=false,bookmarksopen=false,
breaklinks=false,pdfborder={0 0 0},pdfborderstyle={},backref=false,colorlinks=false}
\fi
\usepackage{breakurl}
\usepackage{color}
\usepackage{times}
\usepackage{xcolor}
\usepackage{listings}
\lstset{
language=R,
keywordstyle=\color{blue!70}\bfseries,
basicstyle=\ttfamily,
commentstyle=\ttfamily,
showspaces=false,
showtabs=false,
frame=shadowbox,
rulesepcolor=\color{red!20!green!20!blue!20},
breaklines=true}

\makeatletter
\newcommand{\rmnum}[1]{\romannumeral #1}
\newcommand{\Rmnum}[1]{\expandafter\@slowromancap\romannumeral #1@}
\makeatother

\setlength{\parskip}{.5em}
\renewcommand\arraystretch{1.3}

\title[BI476]{BI476: Biostatistics - Case Studies}
\subtitle[trial]{Lec03: Designing Clinical Trials (临床试验设计)}
\author[Maoying Wu]{Maoying,Wu\\{\small ricket.woo@gmail.com}}
\institute[CBB] % (optional, but mostly needed)
{
  \inst{}
  Dept. of Bioinformatics \& Biostatistics\\
  Shanghai Jiao Tong University
}
\date{Spring, 2018}

\AtBeginSection[]
{
  \begin{frame}<beamer>{Next Section ...}
    \tableofcontents[currentsection]
  \end{frame}
}

\begin{document}
\begin{CJK*}{UTF8}{gbsn}

\frame{\titlepage}

\begin{frame}
\frametitle{Outline}
\tableofcontents
\end{frame}

\section{Clinical Trials: Introduction}

\begin{frame}[t]
\frametitle{Pharmaceutical Drug Design: 3 Phases}
\framesubtitle{Phase I trials: Clinical pharmacology and toxicity study}
\begin{table}
\begin{tabular}{p{0.3\textwidth}p{0.6\textwidth}}
\textbf{Objective:} & The main objective is safety, by providing information on the 
	\textbf{pharmacokinetics} and \textbf{pharmacodynamics}\\
\textbf{Design} & Usually single or multiple dose-escalation studies.\\
\textbf{Subjects} & Normal healthy subjects. Patients may be used, particularly with 
	anti-oncology drugs. The objective in oncology studies is to determine the 
	dose to be used in phase \Rmnum{2} studies (the maximum tolerated dose, MTD)\\ 
\textbf{Sample size} & Approximately 20 to 80 subjects.
\end{tabular}
\uncover<2->{\begin{description}
	\item[Pharmacokinetics (药代)]{Process by which a drug is absored, distributed, metabolized, and eliminated by the body.}
	\item[Pharmacodynamics (药效)]{Study of action or effects of drugs on living organisms or living systems.}
\end{description}}
\end{table}
\end{frame}


\begin{frame}[t]
\frametitle{Pharmaceutical Drug Design: 3 Phases}
\framesubtitle{Phase II trials: Initial clinical investigation on treatment effect}
\begin{table}
\footnotesize
\begin{tabular}{p{0.25\textwidth}p{0.70\textwidth}}
\textbf{Objective:} & To evaluate the potential effectiveness of a drug based on clinical endpoints for a particular indication or indications, the common short-term side effects, and the risks associated with the drug. Providing data on the doses to be used for \Rmnum{3} trials.\\
\textbf{Design: } & Often single-arm, to be compared with historical controls or current treatment:
\begin{itemize}
	\item Randomized dose ranging design
	\item Randomized titration design
	\item Two-stage phase \Rmnum{2} design (oncology)
	\item Multistage design
	\item Bayesian design
	\item Randomized phase \Rmnum{2}
	\item Multiple-endpoint design
\end{itemize}\\
\textbf{Subjects:} & Patients with disease.\\
\textbf{Sample size: } & Often $<100$ patients.
\end{tabular}
\end{table}
\end{frame}

\begin{frame}[t]
\frametitle{Pharmaceutical Drug Design: 3 Phases}
\framesubtitle{Phase III trials: Full-scale evaluation of the effects}
\begin{table}
\begin{tabular}{p{0.25\textwidth}p{0.70\textwidth}}
\textbf{Objective:} & To compare the efficacy of the new treatment with the standard regimen in a scientifically rigorous manner.\\
\textbf{Design: } & Randomized trial of the new treatment \emph{versus} the control regimen.\\
\textbf{Subjects:} & Patients with disease.\\
\textbf{Sample size: } & Often $100$ to $1000+$ patients.
\end{tabular}
\end{table}
We will focus on the phase \Rmnum{3} study in the following slides.
\end{frame}


\begin{frame}[t]
\frametitle{Phase III Clinical Trials}
\framesubtitle{Randomized controlled trials}
\begin{itemize}
	\item Gold-standard clinical design:
	\begin{itemize}
		\item \textbf{New} intervention compared with a \textbf{control}
		\item Treatment assignment made randomly.
	\end{itemize}
	\item Randomization:
	\begin{itemize}
		\item Remove \textbf{bias} in subject allocation to treatments.
		\item Tends to produce comparable group (w.r.t. confounders)
		\item Allows valid statistical tests to be performed.	
	\end{itemize}
\end{itemize}
Randomized clinical trials (RCTs) are the standard to which other designs are compared.
\end{frame}

\begin{frame}[t]
\frametitle{Clinical Trials: Notations and Priciples}
A clinical trial is a \alert{propspective study} comparing the \alert{efficacy and safety} 
of an \alert{intervention (干预)} against a \alert{control (对照)} in human subjects.

\uncover<2->{\begin{alertblock}{\center 3 Principles}
\begin{itemize}
	\item Randomization (随机)
	\item Controlled (对照)
	\item Replication/Sample size (重复/样本量)
\end{itemize}
\end{alertblock}}

\uncover<3->{The control can be 
\begin{itemize}
	\item<4-> external/historical control (外部/历史对照)
	\item<5-> no-treatment control (空白对照)
	\item<6-> placebo control (安慰剂对照)
	\item<7-> active/positive control (阳性对照)
	\item<8-> dose-response control (剂量-响应对照)
	\item<9-> hybrid control (组合对照)
\end{itemize}}
\end{frame}


\begin{frame}[t]
\frametitle{A Randomized Clinical Trial: Example}
{\large \alert{Effect of Nitroglycerin Ointment on Bone Density
and Strength in Postmenopausal Women}}\\
\small{Sophie A. Jamal, et al. JAMA 2011; 305(8):800-807}
\begin{description}
	\item[Objective]To determine if nitroglycerin increases 
		lumbar spine bone mineral density (BMD)
	\item[Design]\underline{\bf Single-center}, \underline{\bf double-blind}, \underline{\bf placebo-controlled} 
		\underline{\bf randomized} trial.
	\item[Intervention]Nitroglycerin ointment (15 mg/d) or placebo 
		applied at bedtime for 24 months.
	\item[Primary outome]Areal BMD at the lumbar spine, femoral neck, 
		and total hip.
	\item[Results]blahblah
	\item[Conclusion]Among postmenopausal women, nitroglycerin ointment 
		modestly increased BMD and decreased bone reabsorption.
\end{description}
\end{frame}


\begin{frame}[t]
\frametitle{A Randomized Clinical Trial: Example}
\framesubtitle{Participant flow diagram}
\begin{figure}
\includegraphics[width=0.5\textwidth]{images/nitroglycerin_bmd_trial_flowchart.png}
\end{figure}
\end{frame}

\begin{frame}[t]
\frametitle{Nitroglycerin and BMD: Baseline characteristics}
\begin{table}
\footnotesize
\caption{Baseline characteristics of study participants}
\begin{tabular}{lcc}
\hline
Characteristics & Nitroglycerin (n=126) & Placebo (n=117)\\
\hline
Age (y) & 61.3 (6.6) & 61.9 (7.3)\\
\hline
Weight (kg) & 70.3 (11.9) & 70.9 (13.3)\\
\hline
White race (\%) & 118 (94) & 107 (91)\\
\hline
Years since menopause & 11.8 (8.2) & 11.8 (8.3)\\
\hline
Walks $\ge 2h$ per wk (\%) & 104 (89) & 109 (87)\\
\hline
Nonsmoker (\%) & 124 (98) & 113 (97)\\
\hline
Vitamine D intake (IU/d) & 783.2 (251.2) & 753.2 (237.2)\\
\hline
Cacium intake (mg/d) & 1548.8 (317.2) & 1565.6 (373.6)\\
\hline
T score (Lumbar spine) & -0.9 (0.6) & -1.1 (0.6)\\
\hline
T score (Femoral neck) & -0.9 (0.6) & -0.8 (0.7)\\
\hline
T score (Total hip) & -0.6 (0.7) & -0.6 (0.7)\\
\hline  
\end{tabular}
\end{table}
\end{frame}

\begin{frame}[t]
\frametitle{Results}
\begin{table}
\footnotesize
\caption{Absolute BMDs of different sites at baseline, 12 and 24 months}
\begin{tabular}{lccc}
\hline
 & \multicolumn{3}{c}{BMD, Absolute value (95\%CI)}\\
 \cline{2-4}
{\bf Site and Group} & {\bf Baseline} & {\bf 12 months} & {\bf 24 months}\\
\hline
Lumbar spine & & & \\ 
- Placebo & 1.06 (1.05-1.08) & 1.06 (1.05-1.08) & 1.08 (1.08-1.09)\\
- Nitroglycerin & 1.05 (1.04-1.07) & 1.11 (1.10-1.13) & 1.14 (1.12-1.15)\\
\hline
Total hip & & & \\
- Placebo & 0.93 (0.91-0.94) & 0.92 (0.91-0.94) & 0.92 (0.90-0.94)\\
- Nitroglycerin & 0.92 (0.91-0.94) & 0.96 (0.94-0.98) & 0.97 (0.96-0.99)\\
\hline
Femoral neck & & & \\
- Placebo & 0.87 (0.86-0.89) & 0.87 (0.85-0.88) & 0.86 (0.85-0.88)\\
- Nitroglycerin & 0.88 (0.86-0.90) & 0.91 (0.89-0.92) & 0.93 (0.92-0.95)\\
\hline
\end{tabular}
\end{table}
\end{frame}



\begin{frame}[t]
\frametitle{Sample size requirements}
In order to compute the sample size required for comparing two means, we need
\begin{itemize}
	\item $\delta_0 = 0.02$: difference between the nitroglycerin and control group;
	\item $\sigma = 0.045$: estimated standard deviation for treatment or control group;
	\item $\alpha = 0.05$: Two-tailed type I error;
	\item $1-\beta = 0.90$: Power to detect the difference
\end{itemize}

Then, sample size for each group is:
$$
\begin{aligned}
n &=& 2 \times \left( \frac{(z_{1-\frac{\alpha}{2}} + z_{1-\beta})\sigma}{\delta_0}\right)^2\\
 &=& 106.4 \approx 107
\end{aligned}
$$ 
\end{frame}

\begin{frame}[t]
\frametitle{Statistical test for comparing the differences in changes of BMD}
\begin{itemize}
	\item $H_0: \mu_1 = \mu_2$;
	\item $H_a: \mu_1 \neq \mu_2$;
	\item Two side $t$-test with $\alpha = 0.05$;
	\item Statistic: $t = \frac{\bar{x}_1 - \bar{x}_2}{S_{\bar{x}_1 - \bar{x}_2}}$
	\item $S_{\bar{x}_1 - \bar{x}_2} = \sqrt{S_c^2 \frac{n_1 + n_2}{n_1 n_2}}$
	\item $S_c^2 = \frac{(n_1-1)S_1^2 + (n_2 - 1)S_2^2}{(n_1-1)+(n_2-1)}$
	\item $t = 8.80 \sim t_{df=n_1+n_2-2}$
	\item $p = 2.69 \times 10^{-16}$
	\item Reject the null hypothesis.
	\item $95\%$ confidence interval of difference: $(\bar{x}_1 - \bar{x}_2) \pm t_{0.975, n_1+n_2-2} 
		\times S_{\bar{x}_1 - \bar{x}_2}$
\end{itemize}
\end{frame}


\begin{frame}[t]
\frametitle{Key issues and the corresponding solution}
\begin{table}
\footnotesize
\caption{Issues and Solutions}
\begin{tabular}{p{0.3\textwidth}p{0.6\textwidth}}
\hline
\textbf{Issues} & \textbf{Solutions}\\
\hline
Procedure selection bias & \alert{Randomization (随机化)}\\
\hline
Assessment bias & \alert{Masking/blinding (盲法)}\\
\hline
Assessment bias & Objective assessment (客观评价)\\
\hline
Treatment-time confounding & Concurrent controls (同期对照)\\
\hline
Disease remission/progression & Concurrent controls (同期对照)\\
\hline
Variation & \alert{Replication (重复,保证足够样本量)}\\
\hline
\end{tabular}
\end{table}
\end{frame}


\section{Clinical trial designs}

\begin{frame}[t]
\frametitle{Clinical Trials: Various Controls}
\begin{itemize}
	\item No control
	\item Historical control (single-arm)
	\item Concurrent but non-randomized control
	\item Randomized
\end{itemize}
\end{frame}

\begin{frame}[t]
\frametitle{Historical control design}
A.k.a \alert{single-arm study (单臂试验)}.

In trials with historical controls, a new treatment is used in a series of subjects; the outcome 
is compared with previous series of comparable subjects.
\begin{alertblock}{\center Pros}
\begin{itemize}
	\item Rapid, inexpensive, good for initial testing of new treatment 
\end{itemize}
\end{alertblock}
\begin{alertblock}{\center Cons: Vulnerable to bias}
Changes in outcome over time may come from:
\begin{itemize}
	\item change in underlying patient population
	\item change in criteria for selecting patients
	\item change in patient care and management peripheral to treatment
	\item change in diagnostic or evaluating criteria
	\item change in quality of data available
\end{itemize}
Studies with historical controls tend to exaggerate the value of new treatment.

Control groups taken from the literature are a particularly poor choice.

Covariate analysis can be used to adjust for patient selection, but all other 
biases will remain. 
\end{alertblock}
\end{frame}


\begin{frame}[t]
\frametitle{Concurrent control design (同期对照)}
\begin{itemize}
	\item Not randomized
	\item Patients compared, treated by different strategies, same period
\end{itemize}

\begin{alertblock}{\center Advantage}
\begin{itemize}
	\item Eliminate time trend
	\item Data of comparable quality
\end{itemize}
\end{alertblock}

\begin{alertblock}{Disadvantage}
\begin{itemize}
	\item Selection Bias
	\item Treatment groups not comparable
	\item Covariance analysis not adequate
\end{itemize}
\end{alertblock}
\end{frame}

\begin{frame}[t]
\frametitle{Randomization can reduce the selection bias}
\begin{table}
\footnotesize
\caption{Clinical trials on the use of anticoagulant therapy on acute MI (1977)}
\begin{tabular}{p{0.20\textwidth}|p{0.20\textwidth}p{0.20\textwidth}p{0.20\textwidth}}
\hline
 & \#studies & \#($p<.05$) & Estimated reduction in total mortality\\
\hline
Non-randomized & & & \\
- Historical controls & 18 & 15 & 50\%\\
- Concurrent controls & 8 & 5 & 50\%\\
\hline
Randomized & 6 & 1 & 20\%\\
\hline
\end{tabular}
\end{table}
\begin{itemize}
	\item<2-> The difference in estimated reduction is probably due to biases in 
		the non-randomized trials.
	\item<3-> Selection bias can lead historically controlled studies to inappropriately 
		favor the new intervention.
	\item<4-> However, small sample sizes in randomized trials lead to missing benefits 
		of new treatments that truly exists. 
\end{itemize}
\end{frame}

\begin{frame}[t]
\frametitle{Commonly used Phase-III designs}
\begin{itemize}
	\item Parallel (平行)
	\item Crossover (交叉)
	\item Factorial (因子)
	\item Group/Cluster (组)
	\item Adaptive (适应性)
\end{itemize}
\end{frame}

\begin{frame}[t]
\frametitle{Parallel design}
\begin{table}
\begin{tabular}{lc}
\hline
\textbf{Patient} & \textbf{Treatment}\\
\hline
1 & A\\
2 & B\\
3 & B\\
4 & A\\
\hline
\end{tabular}
\end{table}
\begin{itemize}
	\item In a parallel study design, each subject is randomized to one 
		and only one treatment.
	\item Most large clinical trials adopt this approach.
	\item During the trial, participants in one group receive drug A \alert{
		in parallel} to participants in the other group receiving drug B
\end{itemize}
\end{frame}


\begin{frame}[t]
\frametitle{Crossover design}
\begin{table}
\begin{tabular}{lcc}
\hline
\textbf{Patient} & \textbf{Period 1} & \textbf{Period 2}\\
\hline
1 & A & B\\
2 & B & A\\
\hline
\end{tabular}
\end{table}
Each patient is randomly assigned to "A, then B" or "B, then A".
\uncover<2->{
\begin{alertblock}{Pros}
\begin{itemize}
	\item Each subject serves as own control $\Rightarrow$ variability reduced.
\end{itemize}
\end{alertblock}}
\uncover<3->{\begin{alertblock}{Cons}
\begin{itemize}
	\item Condition must be chronic (e.g., HBP, arthritis), either "cure" or "death" 
		before the second treatment would ruin the design.
\end{itemize}
\end{alertblock}}

\uncover<4->{This design must assume \alert{no carryover (residual) effect} of the first treatment.
The statistical test for carryover has low power. Else you need \alert{wash-out} period to avoid a \alert{carry-over} effect.}  
\end{frame}

\begin{frame}[t]
\frametitle{Crossover Trials: Advantage and disadvantages}
\uncover<1->{\begin{alertblock}{\center Advantages}
\begin{itemize}
	\item Reducing the variability since the treatment comparison is 
		only \alert{within-subject} other than between-subject
	\item Smaller sample size needed. 
\end{itemize}
\end{alertblock}}

\uncover<2->{\begin{alertblock}{\center Disadvantages}
\begin{itemize}
	\item Strict assumption about carry-over effects;
	\item Only appropriate for chronic diseases;
	\item Drop-out may occurred before second period;
	\item Period effect.
\end{itemize}
\end{alertblock}}
\end{frame}

\begin{frame}[t]
\frametitle{High-order Crossover Designs}
\begin{itemize}
	\item Note that in crossover design, the number of periods does not necessarily have to be 
		equal to the number of treatments to be compared.
	\item Here is an example of $2 \times 3$ crossover design for comparing two treatments with 
		three periods.
\end{itemize}
\uncover<2->{\begin{table}
\footnotesize
\caption{Two-sequence Dual Crossover Design}
\begin{tabular}{lccc}
\hline
 & \textbf{Period 1} & \textbf{Period 2} & \textbf{Period 3}\\
\hline
Sequence 1 & A & B & B\\
Sequence 2 & B & A & A\\ 
\hline
\end{tabular}
\end{table}}
\uncover<3->{\begin{alertblock}{Types of high-order crossover designs}
\begin{itemize}
	\item \alert{Balaam's design}: AA, BB, AB, BA
	\item \alert{Two-sequence dual design}: ABB, BAA
	\item \alert{Double (replicated) design}: AABB, BBAA
	\item \alert{Four-sequence design}: AABB, BBAA, ABBA, BAAB
	\item \alert{William's design with three treatments}: ACB, BAC, CBA, BCA, CAB, ABC
	\item \alert{William's design with four treatments}: ADBC, BACD, CBDA, DCAB
\end{itemize}
\end{alertblock}}
\end{frame}

\begin{frame}[t]
\frametitle{Factorial design}
Assumed that there are two different treatments: $A$ and $B$
\begin{table}
\begin{tabular}{lcc}
& Treatment $B$ & Control\\
\hline
Treatment $A$ & $A+B$ & $A$\\
\hline
Control & $B$ & Control\\
\hline
\end{tabular}
\end{table}
Randomization of subjects to one of 4 possible regimens.

\uncover<2->{\begin{alertblock}{Pros}
\begin{itemize}
	\item Conduct 2 experiments at once!
	\item Investigate potential interaction between $A$ and $B$
\end{itemize}
\end{alertblock}}
\uncover<3->{\begin{alertblock}{Cons}
\begin{itemize}
	\item If \textbf{interaction exists}, each treatment must be tested separately within 
		each level of the other ($\Rightarrow$ reduced power).
\end{itemize}
\end{alertblock}}

\uncover<4->{But if there are three dffernet treatments?}\\
\uncover<5->{\alert{Balanced $2\times 2 \times 2$ factorial design.}}
\end{frame}


\begin{frame}[t]
\frametitle{Group allocation design}
Groups (clinics, communities) are randomized to treatment or control (e.g. 
trials on fluoridated water).
\begin{alertblock}{Pros}
\begin{itemize}
	\item Sometimes logistically more feasible.
	\item Avoids individual consent problem.
\end{itemize}
\end{alertblock}
\begin{alertblock}{Cons}
\begin{itemize}
	\item Many units must participate to overcome unit-to-unit variation.
	\item Larger sample size required than simple randomized design.
\end{itemize}
\end{alertblock}
\end{frame}


\begin{frame}[t]
\frametitle{Cluster RCT Example}
\framesubtitle{1. Population settings}
\footnotesize
\textbf{Aim of study}\\
To examine the effectiveness of a school intervention for well-being and 
health risk behaviors.

\textbf{Study design}\\
Cluster RCT

\textbf{Source population}\\
Metropolitan Melbourne and rural districts, Australia

\textbf{Study year}\\
1997

\textbf{Eligible population}\\
Schools in 12 districts in two education regions in Melbourne and schools 
in 4 rural districts.

\textbf{Selected population}\\
26 metropolitan government, independent and catholic schools and country 
schools.

\textbf{Age}\\
13-14 years (year 8)

\textbf{Female}\\
53.2\%

\textbf{Excluded population}\\
Classrooms in government, independent and catholic metropolitan schools 
and country schools. 
\end{frame}


\begin{frame}[t]
\frametitle{Cluster RCT Example}
\framesubtitle{2. Intervention allocation method}
\footnotesize
Demographic factors.

\textbf{Intervention/s}\\
Intervention involved institutional and individual-based components based 
on an understanding of mental health and risk behaviors that derive from 
social environments.

On a whole school level, intervention involved establishing an 
"adolescent health team" to identify effective strategies to address risk 
issues.

The teaching part of the intervention was derived over 10 weeks in 2 
school years (years 8 and 9).

\textbf{Intervention category}\\
School-based

\textbf{Intervention period}\\
10 weeks during 2 years

\textbf{Control/s}\\
No intervention.

\textbf{Sample sizes}:
Total $n = 26$ schools, 2678 students.\\
Intervention $n = 12$\\
Control $n=14$

\textbf{Baseline comparisons}\\
The intervention group reported slightly lower-level of parental smoking 
and parental separation.   
\end{frame}

\begin{frame}[t]
\frametitle{Cluster RCT Example}
\framesubtitle{3. Outcome and methods of analysis}
\footnotesize
\textbf{Primary outcomes}\\
Smoking Prevalence (any smoking or regular smoker)

\textbf{Secondary outcomes}\\
None

\textbf{Follow-up periods}\\
1,2 and 3 years from baseline.

\textbf{Evaluation}\\
Students completed questionnaires at baseline (beginning of year 8) and 
were followed-up at 1 (end of year 8), 2 (end of year 9) and 3 years (end 
of year 10). Absent students were surveyed at a later date or telephoned 
(along with students who had left the schools).

\textbf{Analysis method}\\
Multivariate analysis. Stated that analysis was intention-to-treat but it 
appears that only students that took part in each measurement stage were 
included in the analysis.
\end{frame}


\begin{frame}[t]
\frametitle{Cluster RCT Example}
\framesubtitle{4. Results}
\footnotesize
\textbf{Primary outcomes}\\
\begin{table}
\caption{Prevalence of smoking (intervention vs. control)}
\begin{tabular}{lp{0.2\textwidth}p{0.2\textwidth}p{0.2\textwidth}}
\hline
  & Year 1 & Year 2 & Year 3\\
\hline
Any smoking & 22.0\% vs. 24.9\% (OR: 0.89 (0.72-1.12)) & 25.0\% vs. 18.7\% (OR: 0.92 (0.63-1.33)) & 24.9\% vs. 28.2\% (OR: 0.91 (0.67-1.24))\\
\hline 
Regular smoking & 4.9\% vs. 8.3\% (OR: 0.66 (0.46-0.95)) & 7.7\% vs. 11.9\% (OR: 0.72 (0.47-1.09)) & 11.8\% vs. 15.6\% (OR: 0.79 (0.58-1.07))\\
\hline
\end{tabular}
\end{table}
Reported ORs were adjusted for baseline measurements and gender, family 
structure, Australian born and parental structure.
\end{frame}

\begin{frame}[t]
\frametitle{Cluster RCT Example}
\framesubtitle{5. Notes}
\footnotesize
\textbf{Limitations identified by author}\\
The small number of schools in the trial limits the effectiveness of the 
randomization process.

\textbf{Limitations identified by review team}\\
Although it is implied taht schools were the units of randomization, 
randomization was primarily by district. It is unclear whether this was 
taken into account in the analysis.

\textbf{Evidence gaps and/or recommendations for future research}\\
Research to investigate specific mechanisms that affect change.


\end{frame}

\begin{frame}[t]
\frametitle{Hybrid design}
\textbf{Hybrid design} combines historical controls and traditional controls.

These \textbf{criteria} must be met:
\begin{itemize}
	\item Same entry criteria and evaluation factors.
	\item Participant recruitment by the same clinic or investigator
	\item Data from historical control participants must be fairly current
\end{itemize}
\begin{alertblock}{Advantages}
\begin{itemize}
	\item Potentially, the need for few participants to be entered into a trial.
\end{itemize}
\end{alertblock}
\begin{alertblock}{Disadvantages}
\begin{itemize}
	\item Bias can be introduced from nonrandomized participants (historical controls).
\end{itemize}
\end{alertblock}
\end{frame}


\begin{frame}[t]
\frametitle{Adaptive design}
\textbf{Adaptive trial design} refers to a clinical trial methodology that allows trial design 
modifications to be made after patients have been enrolled in a study, without compromising the 
scientific method.

In order to maintain the integrity of the trial, these modification should be \alert{clearly 
pre-defined} in the protocol. When designed well, an adaptive trial empowers sponsors to respond 
to data collected during the trial.

\uncover<2->{\begin{alertblock}{\center Types of adaptive trial designs}
\begin{itemize} 
	\item Dropping a treatment arm
	\item Modifying the sample size
	\item Balancing treatment assignment using an adaptive randomization
	\item Stopping a study early for success or failure
\end{itemize}
\end{alertblock}}
\end{frame}


\section{Efficacy Assessment}

\begin{frame}[t]
\frametitle{Efficacy of the new treatment compared to a control (有效性评估分类)}
\begin{itemize}
	\item The new treatment has superior efficacy to the control (Superiority trial, 优效性试验)
	\item The new treatment has efficacy equivalent to that of the active control (Equivalence trial,等效性试验)
	\item The new treatment is not much worse than the active control (Non-inferiority trial, 非劣效性试验)
\end{itemize}
\end{frame}



\begin{frame}[t]
\frametitle{Superiority Design (优效性试验设计)}
\textbf{Aim}\\
Show that a new drug is \textbf{better} than control w.r.t. the efficacy variable of interest.

\textbf{Statistical Tests}\\
$H_0$: No difference in effect between the treatment and controls.

\begin{table}
\caption{Superiority trials}
\begin{tabular}{p{0.20\textwidth}p{0.25\textwidth}p{0.25\textwidth}}
\hline
$p$-value & Indication & Conclusion\\
\hline
$p<.001$ & Strong evidence & "is superior"\\
$p=0.02$ & Some evidence & "Seems superior"\\
$p=0.06$ & Weak evidence & "Might be superior"\\
$p=0.3$ & No evidence & "Seems not superior"\\
\hline
\end{tabular}
\end{table}
\end{frame}


\begin{frame}[t]
\frametitle{Equivalence Design (等效性试验设计)}
\textbf{Purpose}\\
To confirm the \alert{absence of a clinically meaningful difference} between 
treatments.

\textbf{Hypothesis testing}
Equivalence is inferred when \alert{ENTIRE confidence interval falls exclusively 
within equivalence margin: }
$$
(-\delta, +\delta)
$$
\end{frame}


\begin{frame}[t]
\frametitle{Noninferiority Design (非劣效性试验设计)}
A non-inferiority trial aims to demonstrate that the effect of a new treatment 
is as good as, or better than, that of the standard one.

This is assessed by demonstrating that the new treatment is \alert{not worse than} 
the comparator by more than a specified margin ($\delta_0$).

The new treatment might be tested to establish taht it matches the efficacy of 
standard one, and meanwhile has secondary advantages (e.g., in terms of safety, 
convenience to the patients, or cost-effectiveness).

Alternatively, it might have potential as a \alert{second-line therapy} to the 
standard (in cases when the standard fails or is not tolerated).
\end{frame}


\begin{frame}[t]
\frametitle{Hypothesis formula}
\begin{table}
\footnotesize
\caption{Hypotheses formulation for superiority, non-inferiority, and equivalence trials}
\begin{tabular}{llll}
\hline
Study type & Null hypothesis & Alternative hypothesis & Statistic\\
\hline
Statistical superiority & $H_0: C - T \ge 0$ & $H_a: C - T < 0$ & $Z=\delta/s$\\
Clinical superiority & $H_0: C - T \ge -\delta_0$ & $H_a: C - T < -\delta$ & $Z=(\delta - \delta_0)/s$\\
Non-inferiority & $H_0: C - T \ge \delta_0$ & $H_a: C - T < \delta$ & $Z=(\delta + \delta_0)/s$\\
Equivalence & $H_0: |C - T| \ge \delta_0$ & $H_a: |C - T| < \delta$ & $Z_1 = (\delta+\delta_0)/s, Z_2=(\delta-\delta_0)/s$\\
\hline
\end{tabular}
\end{table}
\begin{itemize}
	\item $C$: control or standard treatment;
	\item $T$: new treatment;
	\item $\delta_0$: clinically admissible margin of non-inferiority/
		equivalence/superiority;
	\item $\delta$: Observed difference;
	\item One-sided test is performed in both superiority and non-inferiority 
		trials;
	\item Two-sided test is performed in equivalence test.
\end{itemize}
\end{frame}


\begin{frame}[t]
\frametitle{Hypothesis testing: Sample size for continuous outcome}
\only<1>{\begin{figure}
\includegraphics[width=0.50\textwidth]{images/three_testings.png}
\end{figure}}
\begin{itemize}
	\item<2-> \alert{Non-inferiority design}
	$$
N = 2 \times \left( \frac{z_{1-\alpha} + z_{1-\beta}}{\delta_0} \right)^2 \times s^2
	$$
	\item<3-> \alert{Equivalence design}
	$$
N = 2 \times \left( \frac{z_{1-\frac{\alpha}{2}} + z_{1-\beta}}{\delta_0} \right)^2 \times s^2
	$$
	\item<4-> \alert{Statistical superiority design}
	$$
N = 2 \times \left( \frac{z_{1-\frac{\alpha}{2}} + z_{1-\beta}}{\delta} \right)^2 \times s^2
	$$
	\item<5-> \alert{Clinical superiority design}
	$$
	N = 2 \times \left( \frac{z_{1-\frac{\alpha}{2}} + z_{1-\beta}}{\delta - \delta_0} \right)^2 \times s^2
	$$
\end{itemize}
\end{frame}


\begin{frame}[t]
\frametitle{Hypothesis testing: Sample size for dichotomous outcome}
\only<1>{\begin{figure}
\includegraphics[width=0.50\textwidth]{images/three_testings.png}
\end{figure}}
\begin{itemize}
	\item<2-> \alert{Non-inferiority design}
	$$
N = 2 \times \left( \frac{z_{1-\alpha} + z_{1-\beta}}{\delta_0} \right)^2 \times p \times (1-p)
	$$
	\item<3-> \alert{Equivalence design}
	$$
N = 2 \times \left( \frac{z_{1-\frac{\alpha}{2}} + z_{1-\beta}}{\delta_0} \right)^2 \times p \times (1-p)
	$$
	\item<4-> \alert{Statistical superiority design}
	$$
N = \frac{1}{2} \times \left( \frac{z_{\frac{\alpha}{2}} + z_{1-\beta}}{\arcsin \sqrt{p} - \arcsin \sqrt{p_0}} \right)^2
	$$
	\item<5-> \alert{Clinical superiority design}
	$$
	N = 2 \times \left( \frac{z_{1-\alpha} + z_{1-\beta}}{\delta - \delta_0} \right)^2 \times p \times (1-p)
	$$
\end{itemize}
\end{frame}

\begin{frame}[t]
\frametitle{Sample size for dichotomous outcome: Example}
\textbf{Goal}
To test whether there is a difference in the efficacy of mirtazapine (new drug) 
and sertraline (standard drug) for the treatment of resistant depression in 
6-week treatment duration.
\begin{itemize}
	\item $p = 0.40$: the response rate of standard treatment group;
	\item $p_0 = 0.58$: the response rate of new drug treatment group;
	\item $\delta = p_0 - p = 0.18$: the real difference between the two treatment effects;
	\item $\delta_0 = 0.10$: clinically admissible margin;
	\item $1-\beta = 0.80$: Power;
	\item $1-\alpha = 0.95$: Significance level.
\end{itemize}

\uncover<2->{\begin{alertblock}{\center Sample size calculation}
\begin{itemize}
	\item $N_{NI} = 2 \times \left(\frac{z_{0.95} + z_{0.80}}{0.10} \right)^2 \times 0.40 \times 0.60 = 298$
	\item $N_{EQ} = 2 \times \left( \frac{z_{0.975} + z_{0.80}}{0.10} \right)^2 \times 0.40 \times 0.60 = 378$
	\item $N_{SS} = \frac{1}{2} \times \left( \frac{z_{0.025} + z_{0.80}}{\arcsin \sqrt{0.40} - \arcsin \sqrt{0.58}} \right)^2 = 121$
	\item $N_{CS} = 2 \times \left( \frac{z_{0.95} + z_{0.80}}{0.18 - 0.10}\right)^2 \times 0.40 \times 0.60 = 466$
\end{itemize}
\end{alertblock}}
\end{frame}


\section{Randomization techniques}


\begin{frame}[t]
\frametitle{Allocation procedures to achieve balance}
\begin{itemize}[<+->]
	\item Simple randomization (简单随机)
	\item Biased coin randomization (有偏投币随机)
	\item Permuted block randomization (随机排列区块随机)
	\item Balanced permuted block randomization (平衡随机排列区块随机)
	\item Stratified randomization (分层随机)
	\item Minimization method (最小化方法)
\end{itemize}
\end{frame}


\begin{frame}[t]
\frametitle{Simple Randomization}
A specified probability $p$ (usually equal), of patients assigned to 
each treatment arm, remains constant or may change but not a function 
of covariates or response.
\begin{itemize}[<+->]
	\item Fixed-random allocation
	\begin{itemize}
		\item $n$ known in advance, exactly
		\item $n/2$ selected at random and assigned to Trt A, rest to B
	\end{itemize}
	\item Complete randomization (most common)
	\begin{itemize}
		\item $n$ not exactly known
		\item marginal and conditional prob of assignment = 1/2
		\item analogous to a coin-flip/random-digit
	\end{itemize}
\end{itemize}
\end{frame}


\begin{frame}[t]
\frametitle{Restricted Randomization}
\framesubtitle{guarantee of procedure balance (过程平衡)}
\begin{itemize}
	\item Biased coin (Efron)
	\item Urn design (LJ Wei)
	\item Permuted block
\end{itemize}
\end{frame}


\begin{frame}[t]
\frametitle{Biased coin design (BCD)}
\begin{itemize}[<+->]
	\item Allocation probability to Trt A changes to keep balanced in 
		each group nearly equal
	\item $\textrm{BCD}(p)$
	\begin{enumerate}[(1)]
		\item Compute the \alert{running difference} $D = n_A - n_B$ and 
			$n = n_A + n_B$;
		\item Define $p$ to be the probability of assigning Trt $> 1/2$
		\item We can compute $p_A$:
		$$
p_A = \begin{cases}
1/2 & D = 0\\
1-p & D > 0\\
p & D < 0
\end{cases}
		$$
	\end{enumerate}
	\item Efron suggests that $p=2/3$
\end{itemize}
\end{frame}

\begin{frame}[t]
\frametitle{Urn randomization}
\begin{itemize}
	\item Wei \& Lachin: Controlled Clinical Trials, 1988
	\item A generalization of BCD to correct for the constant correction prob 
		(e.g. $2/3$) regardless of the degree of imbalance
	\item Urn design modifies $p$ as the function of the degree of imbalance
	\item $U(n, n)$:
	\begin{enumerate}
		\item Start with Urn containing $n$ white and $n$ red balls;
		\item Ball is drawn at random and replaced;
		\item If red, assign $B$; else assign $A$;
		\item Add 1 ball of opposite color;
		\item Go to 2
	\end{enumerate}
\end{itemize}
\end{frame}


\begin{frame}[t]
\frametitle{Permuted Block Randomization}
\begin{alertblock}{\center Basic Idea}
\begin{enumerate}[(1)]
	\item Divide potential patients into $G$ groups or blocks of size $2m$
	\item Randomize each block such that $m$ patients are allocated to A, and 
		$m$ to B;
	\item The total sample size is $2\times m \times G$
	\item For each block, there are $\binom{2m}{m}$ realizations.
	\item Maximum imbalance at any time is $m$
\end{enumerate}
\end{alertblock}
\end{frame}

\begin{frame}[t]
\frametitle{Permuted Block Randomization: Concerns}
If blocking is NOT masked, the sequence become somewhat predictable 
(e.g., $2m=4$):
\begin{itemize}
	\item ABAB BAB?
	\item AA??
\end{itemize}
This will lead to selection bias.

\uncover<2->{\begin{alertblock}{\center Simple solution}
\begin{itemize}
	\item Do NOT reveal blocking mechanism;
	\item Use random block sizes;
\end{itemize}
\end{alertblock}}
\uncover<3->{\alert{If treatment is double-blinded, no selection 
bias.}}
\end{frame}


\begin{frame}[t]
\frametitle{Balancing on Baseline Covariates (基于基线协变量的平衡)}
\begin{itemize}
	\item Stratified Randomization (分层平衡)
	\begin{itemize}
		\item balanced w.r.t prognostic or risk factors (covariates)
	\end{itemize}
	\item Covariate Adaptive (协变量自适应平衡)
	\begin{itemize}
		\item Minimization
		\item Pocock \& Simon method
	\end{itemize}
\end{itemize}
\end{frame}


\begin{frame}[t]
\frametitle{Stratififed Randomization (分层随机法)}
\begin{itemize}
	\item For large studies, randomization "tends" to be balanced.
	\item For small studies, a better guarantee may be needed.
	\item Divide each risk factor into discrete categories:
	\begin{itemize}
		\item $f$: number of risk factors;
		\item $\ell_i$: number of categories for factor $i$ 
	\end{itemize}
	\item Randomize within each stratum (usually blocked)
\end{itemize}
\uncover<2->{
\begin{table}
\begin{tabular}{lll}
\hline
Age & Male & Female\\
\hline
40- & ABBA, BAAB, ... & BABA, BAAB, ...\\
41-60 & BBAA, ABAB, ... & ABAB, BBAA, ...\\
60+ & AABB, ABBA, ... & BAAB, ABAB, ...\\
\hline
\end{tabular}
\end{table}
}
\end{frame}


\begin{frame}[t]
\frametitle{Minimization}
\begin{itemize}
	\item Balances treatments simultaneously over several prognostic factors (strata)
	\item Does NOT balance \alert{within} cross-classified cells; balance over the 
		\alert{marginal totals} of each stratum separately.
	\item Used when the number of stratum cells is large relative to the sample size 
		(stratified design will yield sparse cells)
	\item Can be computerized 
\end{itemize}
\end{frame}


\begin{frame}[t]
\frametitle{Minimization: Method}
Three stratification factors:
\begin{itemize}
	\item Gender (2 levels)
	\item Age (3 levels)
	\item Disease stage (3 levels)
\end{itemize}
\begin{table}
\footnotesize
\caption{Current assignment for 50 patients}
\begin{tabular}{llcc}
\hline
& & Treatment A & Treatment B\\
\hline
Gender: & Male & 16 & 14\\
& Female & 10 & 10\\
\hline
Age: & 40- & 13 & 12\\
& 41-60 & 9 & 6\\
& 61+ & 4 & 6\\
\hline
Disease stage: & Stage I & 6 & 4\\
& Stage II & 13 & 16\\
& Stage III & 7 & 4\\
\hline
Total & & 26 & 24
\end{tabular}
\end{table}
\end{frame}

\begin{frame}[t]
\frametitle{Minimization}
Say the 51st patient enrolled in the study is male, $age \ge 61$, stage III.
Summarize the above table for the corresponding factors:
\begin{table}
\footnotesize
\begin{tabular}{lccc}
\hline
 & Treatment A & Treatment B & Sign of difference\\
\hline
Male & 16 & 14 & +\\
Age: 61+ & 4 & 6 & -\\
Stage III & 7 & 4 & +\\
\hline
Total & 27 & 24 & 2+,1-
\end{tabular}
\end{table}
\uncover<2->{\begin{alertblock}{\center Two possible criteria}
\begin{itemize}
	\item Count only the direction (sign) of the difference in each factor (\alert{2A vs. 1B}) 
		$\Rightarrow$ \alert{assign the 51st patient to B}.
	\item Add the total overall categories (\alert{27A vs. 24B}) $\Rightarrow$ 
		\alert{assign the 51st patient to B}.
\end{itemize}
\end{alertblock}}
\uncover<3->{The two criteria will usually agree, but not always.}
\end{frame}

\section{Blinding/Masking}

\begin{frame}[t]
\frametitle{Blinding}
The goal of blinding is to minimize the potential biases resulting from differences in
\alert{management, treatment, assessment of patients, interpretation of results}
that could arise as a result of participant or investigator knowledge of assigned treatment.

\uncover<2->{\begin{alertblock}{\center Types of blinding}
\begin{description}
	\item[Single-blind design: ] Only participants are blinded.
	\item[Double-blind design: ] Both participants and investigators are blinded.
	\item[Triple-blind design: ] Participants, investigators, and statisticians 
		are blinded.
	\item[PROBE: ] Prospective Randomized Open with Blinded Endpoint Assessment.
\end{description}
\end{alertblock}}

\uncover<3->{The opposites of a blind trial is \alert{open-label trials}, which are 
		ethical phase I dose-escalating studies in oncology, 
		pre-marketing, post-marketing surveillance.}
\end{frame}


\begin{frame}[t]
\frametitle{Open-label/Unblinded trials}
\begin{figure}
\includegraphics[width=0.40\textwidth]{images/open-label.jpg}
\end{figure}
\uncover<2->{\begin{alertblock}{Advantages}
\begin{itemize}
	\item Simple and fairly inexpensive
	\item A true reflection of clinical practice.
\end{itemize}
\end{alertblock}}
\uncover<3->{\begin{alertblock}{Disadvantages}
\begin{itemize}
	\item Participants may underreport adverse effects of the new treatment.
	\item Investigators might supply different amounts of concomitant treatments (e.g., only giving analgesics to the surgical group).
\end{itemize}
\end{alertblock}}
\end{frame}

\begin{frame}[t]
\frametitle{Single-blinded (单盲) trials}
\framesubtitle{to counteract expectations and the placebo effect}
\begin{itemize}
	\item The participants should be unware of which treatment they are taking.
	\item The investigators are aware of whether the treatment is new, standard 
		or placebo.
\end{itemize}
\only<1>{\begin{figure}
\includegraphics[width=0.65\textwidth]{images/single-blinded.jpg}
\end{figure}}

\uncover<2->{\begin{alertblock}{\center Advantages}
The design is simple and allows investigators to exercise their clinical judgement 
when treating participants.
\end{alertblock}}

\uncover<3->{\begin{alertblock}{\center Disadvantages}
\begin{itemize}
	\item Patients might under- or over-report treatment effects and side-effects, 
		based on some influence or response from the investigators.
	\item Investigators may give advice or prescribe additional therapy to the 
		control arm if they feel that these patients are disadvantaged in 
		comparison to the active arm - bias.
\end{itemize}
\end{alertblock}}
\end{frame}


\begin{frame}[t]
\frametitle{Double-blinded trials}
\begin{figure}
\includegraphics[width=0.40\textwidth]{images/double-blind.jpg}
\end{figure}

\uncover<2->{\begin{alertblock}{\center Advantages}
\begin{itemize}
	\item Reducing the biases incurred by unblindedness.
\end{itemize}
\end{alertblock}}

\uncover<3->{\begin{alertblock}{\center Disadvantages}
\begin{itemize}
	\item Lessening the ability for investigators to monitor the safety of treatments.
\end{itemize}
\end{alertblock}}

\uncover<4->{\begin{alertblock}{\center Limitations}
\begin{itemize}
	\item Matched medication required, especially for crossover design.
	\item Unblinded unless the medication is of identical appearance. 
\end{itemize}
\end{alertblock}}
\end{frame}


\begin{frame}[t]
\frametitle{Triple-blinded trials}
\begin{itemize}
	\item Appropriate for studies with low risk of adverse events;
	\item Not for treatments with critical safety issues.
\end{itemize}
\begin{figure}
\includegraphics[width=0.30\textwidth]{images/triple_blinded.png}
\end{figure}
\uncover<2->{\begin{alertblock}{\center Advantages}
\begin{itemize}
	\item 
\end{itemize}
\end{alertblock}}

\uncover<3->{\begin{alertblock}{\center Disadvantages}
\begin{itemize}
	\item Lessening the chance that the trial may stop early to favor either 
		treatment;
	\item Making the evaluation of results more objective;
	\item Lessening investigator's ability to monitor safety and efficacy.
\end{itemize}
\end{alertblock}}
\end{frame}


\begin{frame}[t]
\frametitle{PROBE trials}
\begin{itemize}
	\item \alert{P}rospective, \alert{R}andomized, \alert{O}pen-label,
		\alert{B}linded \alert{E}ndpoint Design
	\item Phrase coined by Hannsson et al. (Blood Pressure, 1992)
	\item Motivation:
	\begin{itemize}
		\item More similar to clinical practice
		\item Easier to enroll trials
		\item Better patient compliance
		\item Cheaper (?)
	\end{itemize}
\end{itemize}
\begin{alertblock}{\center Example}
\alert{GI-REASONS: a novel 6-month, prospective, randomized, open-label, blinded endpoint (PROBE) trial.}\\
\small{Cryer B1, et al. Am J Gastroenterol. 2013; 108(3):392-400.} 
\end{alertblock}
\end{frame}

\begin{frame}[t]
\frametitle{Blinding: Summary}
\begin{table}
\footnotesize
\caption{Potential benefits of blinding}
\begin{tabular}{p{0.20\textwidth}p{0.65\textwidth}}
\hline
{\bf Blinded individuals} & {\bf Potential benefits}\\
\hline
{\bf Participants} & Less likely to have biased psychological or physical responses to interventions\\
& More likely to comply with trial regimens\\
& Less likely to seek additional adjunct interventions\\
& Less likely to leave trials without providing outcome data (lost-to-follow-up)\\
\hline
{\bf Investigators} & Less likely to transfer their inclination or attitudes to participants\\
& Less likely to differentially administer co-interventions\\
& Less likely to differentially adjust dose.\\
& Less likely to differentially withdraw participants.\\
& Less likely to differentially encourage or discourage participants to continue trial.\\
\hline
{\bf Assessors} & Less likely to have biased affect their outcome assessments, especially with subjective outcomes of interest.\\
\hline 
\end{tabular}
\end{table}
\end{frame}

\section{Intention-to-Treat Analysis}

\begin{frame}[t]
\frametitle{Intention-to-treat (IIT) analysis}
\framesubtitle{Once randomized, always analyzed!}
RCTs often suffer from two major complications: \alert{noncompliance (违反协议)} and 
\alert{missing outcome (数据缺失)}.

Participants in a trial may experience:
\begin{itemize}
	\item Completed the treatment (完成治疗);
	\item Did NOT receive treatment (未接受任何治疗);
	\item Lost-to-follow-up (失访)
	\item Discontinued to treatment (中途退出,有部分治疗结果数据);
	\item Stop treatment early (早期退出,只由很少数据);
	\item Died (死亡);
	\item Received incorrect treatment (依从性不佳)
\end{itemize}
\end{frame}

\begin{frame}[t]
\frametitle{ITT: An example}
\textbf{Patients: }\\
200 patients with cerebrovascular disease

\textbf{Randomization: }
\begin{itemize}
	\item Group $A$: ASA + Surgery, 100 patients
	\item Group $B$: ASA only, 100 patients
\end{itemize}

\textbf{Follow-up: }
1 year.

\textbf{Result: }
\begin{itemize}
	\item 10 patients in group $A$ got stroke before surgery.
	\item 10 patients in group $B$ got stroke within a month.
	\item 10 more patients in group $A$ got stroke after surgery within a year.
	\item 10 more patients in group $B$ got stroke within a year of follow-up.
\end{itemize}

\alert{Any significant difference between the two groups?}
\end{frame}


\begin{frame}[t]
\frametitle{Three common approaches}
\alert{ITT (按治疗意向分析)} analysis includes every subject who is randomized accordign to randomized 
treatment assignment by ignoring noncompliance, protocol deviations, withdrawal, 
and anything that happens after randomization.

\alert{As-Treated (AT) Analysis (按真实治疗分析)} analyze the data according to the real treatment received.

\alert{Per-protocol (PP, 按符合计划分析)} population is defined as a subset of ITT population who 
complete the study without any major protocol violations.

\alert{Modified ITT (mITT)} is a subset of ITT population and allows the exclusion 
of some randomized subjects in a justified way (such as patients who were deemed 
ineligible after randomization or certain patients who never started treatment.

\uncover<2->{\alert{Then, which approach should we choose, for the above example?}}

\begin{itemize}
	\item<3-> Ignoring those getting stroke before surgery in group $A$?
	\item<4-> Transfer those getting stroke before surgery in group $A$ to group $B$?
	\item<5-> Intention-to-treat?
\end{itemize}
\end{frame}


\begin{frame}[t]
\frametitle{RCTs Example}
\framesubtitle{A clinical trial for primary breast cancer (1972)}
\textbf{Study question}\\
Does L-Pam prolong the disease-free interval of primary breast cancer patients after radical mastectomy?

\textbf{Randomized treatments}\\
\begin{itemize}
	\item L-Pam (orally, 0.15mg/kg body weight, 5 consecutive days every 6 weeks for 2 years, with 
		specified dose modification for hematological toxicity)
	\item Placebo (physically indistinguishable from L-Pam)
\end{itemize}

\textbf{Eligibility}
\begin{itemize}
	\item Radical mastectomy for primary breast cancer (with 4 weeks of starting protocol treatment)
	\item Historically confirmed axillary node involvement.
	\item No skin ulceration or peau d'orange
	\item Age $\le$ 75 years
	\item Not pregnant or lactating
\end{itemize}
Focus is on those patients most likely to benefit.
\end{frame}


\begin{frame}[t]
\frametitle{RCTs Example}
\framesubtitle{A clinical trial for primary breast cancer (1972)}
\textbf{Outcomes}
\begin{itemize}
	\item Primary outcome: disease-free interval = time from mastectomy till fist detection of tumor (local 
		regional, or distance)
	\item Other outcomes:
	\begin{itemize}
		\item Survial time = time from mastectomy until death
		\item Toxicity = occurrences of hematological toxicity or nausea/vomting
	\end{itemize}
\end{itemize}
\end{frame}

\begin{frame}[t]
\frametitle{RCTs Example}
\framesubtitle{A clinical trial for primary breast cancer (1972)}
\textbf{The study}
\begin{itemize}
\small
	\item A \underline{\bf written protocol} documented all study procedures and information. \alert{Sample size 
		requirements} dictated several hundred patients $\Rightarrow$ a \alert{multicenter} trial (37 hospitals) 
		was undertaken.
	\item \underline{\bf Randomization} was performed by phoning the central office. Patients were \alert{stratified} 
		by age ($<50$, $\ge 50$), nodal status (1 to 3 vs. 4+ positive axillary nodes), and institution.
	\item The trial was \alert{double-blinded}
	\item \underline{\bf Follow-up} exams were performed every 6 weeks, with test for hematological toxicity every 3 
		weeks $\Rightarrow$ outcome evaluation was done consistently and objectively.
	\item Patient \underline{\bf accrual} started Sep 1972 and ended Feb 1975. In total, 370 were accrued.
	\item \underline{\bf Information consent} was obtained from all patients.
	\item The \textbf{trial committee} reviewed the study regularly. After the first few months they decided 
		to relax eligibility w.r.t. \#(positive nodes)
	\item The \textbf{central coordinating office} supervised data collection and processing.
	\item As the data accumulated, \underline{\bf interim analyses} were preformed. There was pressure to release 
		interim results
\end{itemize}
\end{frame}

\begin{frame}[t]
\frametitle{RCTs Example}
\framesubtitle{A clinical trial for primary breast cancer (1972)}
\textbf{Results}
\begin{itemize}
	\item The final results showed that L-Pam significantly increased the disease-free 
		survival time ($p=0.009$)
	\item \textbf{Subgroup analysis} revealed that L-Pam had the largest benefit in 
		younger patients ($\le 50$ years old).
	\item \textbf{Toxicity}: Hematological toxicity was common (25\%), but never 
		life-threatening.
	\item \textbf{Side effect}: Nausea/vomting was experienced by 40\% of L-Pam patients 
		and 11\% of placebo patients.
\end{itemize}
\textbf{Conclusion}\\
L-Pam was adopted as the new "standard" treatment.
\end{frame}

\end{CJK*}
\end{document}
