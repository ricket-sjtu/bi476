\documentclass[a4paper,11pt]{article}

\usepackage{CJK}
\usepackage{graphicx}
\usepackage{amsmath,amssymb,amsopn}
\usepackage[english]{babel}
\usepackage[latin1]{inputenc}
\usepackage{url}
\usepackage{enumerate}
\usepackage{multirow}
\usepackage{color}
\usepackage{times}
\usepackage{xcolor}
\usepackage{geometry}
\usepackage{listings}
\lstset{
language=R,
keywordstyle=\color{blue!70}\bfseries,
basicstyle=\ttfamily,
commentstyle=\ttfamily,
showspaces=false,
showtabs=false,
frame=shadowbox,
rulesepcolor=\color{red!20!green!20!blue!20},
breaklines=true
}

\geometry{paperwidth=18.4cm,paperheight=26cm}
\setlength{\parindent}{0em}
\setlength{\parskip}{1.2\baselineskip}

\title{\small{BI476: Biostatistics - Case Studies}\\
	\Large{Exercise 03 - Designing Clinical Trials}
}
\author{Spring, 2018}
\date{}

\begin{document}
\begin{CJK*}{UTF8}{gbsn}
\maketitle

\section{Answer the following questions}
\begin{enumerate}[(1)]
	\item Can we draw a conclusion of equivalence based on the insignificance result 
		of superiority trial? If not, then outline the statistical testing on how 
		to prove that a treatment $T$ is equivalent to $B$ in a parallel trial?
	\item Which test requires a larger sample size for the same $\delta_0$, $\alpha$, 
		power, equivalence trial or non-inferiority trial?
	\item How to deal with the non-compliance of the participants in a clinical trial?
	\item When aren't the double-blind feasible in a clinical trial?
	\item As we have talked about selection bias in the observational study, it is a 
		more severe issue in a randomized controlled trial. Can you use some 
		example to illustrate the types of selection biases in an RCT.
	\item Illustrate how block randomization could be used to randomly allocate 
		treatments to 30 patients with an allocation ratio of 1:2 using a block 
		size of 6.
\end{enumerate}

\section{Data analysis of continuous outcome}
An RCT was conducted to compare two therapies for pain relief after the wisdom tooth 
extraction surgery. A dual-therapy (Acetaminophen+lbuprofen) was compared againt a 
mono-therapy (lbuprofen only). The primary outcome is the post-surgery pain measure 
at an interval of 15 minutes within a follow-up of 120 minutes. The pain was measured 
in a scal of 0 (no pain) to 100 (worst pain).
\begin{table}[t]
\caption{The follow-up pain data after wisdom-tooth extraction surgery}
\begin{tabular}{rccccccc}
\hline
 & \multicolumn{3}{c}{lbuprofen} & &\multicolumn{3}{c}{Acetaminophen+lbuprofen}\\
\cline{2-4} \cline{6-8}
Time (min) & Mean (mm) & S.D. (mm) & N & & Mean (mm) & S.D. (mm) & N\\
\hline
15 & 27.9 & 14.8 & 24 & & 18.2 & 13.1 & 24\\
30 & 32.6 & 24.4 & 25 & & 25.3 & 20.9 & 25\\
45 & 35.5 & 23.2 & 22 & & 28.7 & 23.3 & 20\\
60 & 31.3 & 18.9 & 19 & & 25.1 & 22.8 & 23\\
75 & 29.9 & 18.8 & 24 & & 14.9 & 13.8 & 24\\
90 & 23.8 & 17.9 & 22 & & 15 & 14.2 & 24\\
105 & 22.7 & 16.4 & 21 & & 13.7 & 12.8 & 19\\
120 & 20.9 & 17.2 & 24 & & 15.2 & 14.4 & 23\\
\hline
\end{tabular}
\end{table}
\begin{enumerate}[(1)]
	\item Conduct a $t$-test to compare the treatment effects of the two therapies 
		at each time point. Give also the 95\% confidence interval.
	\item What kind of assumptions should we make when we conduct a $t$-test? So, 
		is it plausible to use $t$-test here? Should we use a nonparametic 
		approach, instead?
	\item We can also calcualte the weighted average of the pain scores across 
		the time points for every patient:
\begin{table}[h]
\begin{tabular}{rccccccc}
\hline
 & \multicolumn{3}{c}{lbuprofen} & &\multicolumn{3}{c}{Acetaminophen+lbuprofen}\\
\cline{2-4} \cline{6-8}
 & Mean (mm) & S.D. (mm) & N & & Mean (mm) & S.D. (mm) & N\\
\hline
summary & 27.9 & 13.6 & 25 & & 19.5 & 12.3 & 26\\
\hline
\end{tabular}
\end{table}
Can you conduct the $t$-test only on the weighted mean to compare the treatment effect.
	\item You can count the number of the tests for all the time points with $p$-value 
		less than 0.05, and then arrive at the final conclusion.
	\item The other approach is to only use the weighted average to reach the conclusion.
		Which one do you prefer? Why?
	\item If the pain reduction $\delta=-8$ is clinically significant, can the sample 
		size in this trial achieve a power of 0.90 to detect such reduction? (Hint: 
		$\alpha=0.05$)
	\item Can you find some flaws for the above study design? Comment.
	\item Try to figure out a method to give the best estimate of the effect of 
		Acetaminophen in pain relief following the wisdom-tooth extraction?
\end{enumerate}


\section{Clinical trial design}
An RCT was comparing the psychological treatment (\textbf{CBT}) with the exercise program 
(\textbf{EX}) for patient suffering from moderate to severe anxiety. Patients are randomized 
to treatment using \textbf{deterministic minimization} controlling for gender and severity. After 
65 patients have entered the trial. The number of patients with each characteristic is summarized 
in the following table:
\begin{table}
\caption{Number of patients for each characteristic}
\center{
\begin{tabular}{llcc}
& & \multicolumn{2}{c}{Treatment}\\
\cline{3-4}
\multicolumn{2}{l}{Characteristics} &  CBT & EX\\
\hline
\multirow{2}{*}{Gender} & Male & 18 & 15\\
& Female & 15 & 17\\
\hline
\multirow{2}{*}{Severity} & Moderate & 22 & 21\\
& Severe & 11 & 11\\
\hline
\end{tabular}}
\end{table}
\begin{enumerate}[(1)]
	\item How many patients have been allocated to each treatment?
	\item If the 66$^{\textrm{th}}$ is a male and moderately severe patient, 
		which treatment would he be allocated?
	\item The 67$^{\textrm{th}}$ is a female and moderately severe patient, 
		which treatment would she be allocated? 
\end{enumerate}



\section{Case study}
Read this article, and answer the following questions.\\
\begin{quote}
Petter Quist-Paulsen. Randomised controlledtrial of smoking cessation 
intervention after admission for coronary heart disease. BMJ 2003;327:1254.
\end{quote}
\begin{enumerate}[(1)]
	\item Using a $z$-test of proportions, check the analysis for Table 2 of 
		the paper to compare the smoking cessation rates in intervention 
		group and control group at 12 months. Report also the $95\%$ 
		confidence interval of the treatment effect.
	\item Summarize the results of the above analysis in your own words.
	\item How does the author deal with the missing data? Since we have talked about 
		intention-to-treat analysis, how does it deal with the missing data? 
		Compare the results with the article.
	\item Comparing the lost-to-follow-up rates between the intervention and 
		control group, what conclusion can you draw from this analysis? 
\end{enumerate}


\section{Case study}
A randomized controlled equivalence trial is conducted to test whether a new \textbf{generic 
drug} is of equal efficacy to the current \textbf{standard drug}. Here is the partial result:
\begin{verbatim}
         | Obs   Mean   Std. Err. Std. Dev. [90% Conf. Interval]
---------+-------------------------------------------------------
Standard | 42   35.2    2.79289   18.1      30.49991    39.90009
Generic  | 41   34.1    2.79551   17.9      29.39278    38.80722
---------+-------------------------------------------------------
\end{verbatim}
The investigators suggest that a difference of 5 is clinically significant. Using the above 
data, tell whether the new generic drug is equivalent to the current standard drug under the 
significance level of $\alpha=0.05$.

\end{CJK*}
\end{document}
