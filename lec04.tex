\documentclass[table,10pt]{beamer}\usepackage[]{graphicx}\usepackage[]{color}
%% maxwidth is the original width if it is less than linewidth
%% otherwise use linewidth (to make sure the graphics do not exceed the margin)
\makeatletter
\def\maxwidth{ %
  \ifdim\Gin@nat@width>\linewidth
    \linewidth
  \else
    \Gin@nat@width
  \fi
}
\makeatother

\definecolor{fgcolor}{rgb}{0.345, 0.345, 0.345}
\newcommand{\hlnum}[1]{\textcolor[rgb]{0.686,0.059,0.569}{#1}}%
\newcommand{\hlstr}[1]{\textcolor[rgb]{0.192,0.494,0.8}{#1}}%
\newcommand{\hlcom}[1]{\textcolor[rgb]{0.678,0.584,0.686}{\textit{#1}}}%
\newcommand{\hlopt}[1]{\textcolor[rgb]{0,0,0}{#1}}%
\newcommand{\hlstd}[1]{\textcolor[rgb]{0.345,0.345,0.345}{#1}}%
\newcommand{\hlkwa}[1]{\textcolor[rgb]{0.161,0.373,0.58}{\textbf{#1}}}%
\newcommand{\hlkwb}[1]{\textcolor[rgb]{0.69,0.353,0.396}{#1}}%
\newcommand{\hlkwc}[1]{\textcolor[rgb]{0.333,0.667,0.333}{#1}}%
\newcommand{\hlkwd}[1]{\textcolor[rgb]{0.737,0.353,0.396}{\textbf{#1}}}%
\let\hlipl\hlkwb

\usepackage{framed}
\makeatletter
\newenvironment{kframe}{%
 \def\at@end@of@kframe{}%
 \ifinner\ifhmode%
  \def\at@end@of@kframe{\end{minipage}}%
  \begin{minipage}{\columnwidth}%
 \fi\fi%
 \def\FrameCommand##1{\hskip\@totalleftmargin \hskip-\fboxsep
 \colorbox{shadecolor}{##1}\hskip-\fboxsep
     % There is no \\@totalrightmargin, so:
     \hskip-\linewidth \hskip-\@totalleftmargin \hskip\columnwidth}%
 \MakeFramed {\advance\hsize-\width
   \@totalleftmargin\z@ \linewidth\hsize
   \@setminipage}}%
 {\par\unskip\endMakeFramed%
 \at@end@of@kframe}
\makeatother

\definecolor{shadecolor}{rgb}{.97, .97, .97}
\definecolor{messagecolor}{rgb}{0, 0, 0}
\definecolor{warningcolor}{rgb}{1, 0, 1}
\definecolor{errorcolor}{rgb}{1, 0, 0}
\newenvironment{knitrout}{}{} % an empty environment to be redefined in TeX

\usepackage{alltt}

\mode<presentation>{
%\usetheme{Goettingen}
\usetheme{Boadilla}
\usecolortheme{default}
}
\usepackage{CJK}
\usepackage{graphicx}
\usepackage{amsmath, amsopn}
\usepackage{xcolor}
\usepackage[english]{babel}
\usepackage[T1]{fontenc}
\usepackage[latin1]{inputenc}
\usepackage{enumerate}
\usepackage{multirow}
\usepackage{url}
\ifx\hypersetup\undefined
	\AtbBeginDocument{%
		\hypersetup{unicode=true,pdfusetitle,
bookmarks=true,bookmarksnumbered=false,bookmarksopen=false,
breaklinks=false,pdfborder={0 0 0},pdfborderstyle={},backref=false,colorlinks=false}
	}
\else
	\hypersetup{unicode=true,pdfusetitle,
bookmarks=true,bookmarksnumbered=false,bookmarksopen=false,
breaklinks=false,pdfborder={0 0 0},pdfborderstyle={},backref=false,colorlinks=false}
\fi
\usepackage{breakurl}
\usepackage{color}
\usepackage{times}
\usepackage{xcolor}
\usepackage{listings}
\lstset{
language=R,
keywordstyle=\color{blue!70}\bfseries,
basicstyle=\ttfamily,
commentstyle=\ttfamily,
showspaces=false,
showtabs=false,
frame=shadowbox,
rulesepcolor=\color{red!20!green!20!blue!20},
breaklines=true}



\setlength{\parskip}{.5em}

\title[BI476]{BI476: Biostatistics - Case Studies}
\subtitle[anova]{Lec04: Clinical Trial Data Analysis}
\author[Maoying Wu]{Maoying,Wu\\{\small ricket.woo@gmail.com}}
\institute[CBB] % (optional, but mostly needed)
{
  \inst{}
  Dept. of Bioinformatics \& Biostatistics\\
  Shanghai Jiao Tong University
}
\date{Spring, 2018}

\AtBeginSection[]
{
  \begin{frame}<beamer>{Next Section ...}
    \tableofcontents[currentsection]
  \end{frame}
}
\IfFileExists{upquote.sty}{\usepackage{upquote}}{}
\begin{document}
\begin{CJK*}{UTF8}{gbsn}



\begin{frame}
  \titlepage
\end{frame}

\begin{frame}
\frametitle{Outline}
	\tableofcontents
\end{frame}

\begin{frame}[t,containsverbatim]
\frametitle{A Clinical Trial on Diastolic Blood Pressure (DBP)}
Here we present a data set of diastolic blood pressure measured in 
small clinical trials in hypertension from the mid-to-late 1960s and 
for approximately a decade thereafter. Diastolic blood pressure (DBP) 
was measured (mmHg) in the supine position at baseline (i.e., ``DBP1'') 
before randomization and monthly thereafter up to 4 months as indicated 
by \texttt{DBP2}, \texttt{DBP3}, \texttt{DBP4} and \texttt{DBP5}. 
Patients' age and sex were recorded at baseline and represent potential 
covariates.

The primary objective in the analysis of this dataset is to test 
whether treatment A (new drug) may be effective in lowering DBP as 
compared to B (placebo) and to describe changes in DBP across the times 
at which it was measured.
\end{frame}


\begin{frame}[t,containsverbatim]
\frametitle{Importing the dataset}
\begin{knitrout}\footnotesize
\definecolor{shadecolor}{rgb}{0.969, 0.969, 0.969}\color{fgcolor}\begin{kframe}
\begin{alltt}
\hlstd{dbp} \hlkwb{<-} \hlkwd{read.table}\hlstd{(}\hlstr{"data/dbp.txt"}\hlstd{,} \hlkwc{header}\hlstd{=T)}
\hlstd{dbp}\hlopt{$}\hlstd{diff} \hlkwb{<-} \hlstd{dbp}\hlopt{$}\hlstd{DBP5} \hlopt{-} \hlstd{dbp}\hlopt{$}\hlstd{DBP1}
\hlkwd{head}\hlstd{(dbp)}
\end{alltt}
\begin{verbatim}
##   Subject TRT DBP1 DBP2 DBP3 DBP4 DBP5 Age Sex diff
## 1       1   A  114  115  113  109  105  43   F   -9
## 2       2   A  116  113  112  103  101  51   M  -15
## 3       3   A  119  115  113  104   98  48   F  -21
## 4       4   A  115  113  112  109  101  42   F  -14
## 5       5   A  116  112  107  104  105  49   M  -11
## 6       6   A  117  112  113  104  102  47   M  -15
\end{verbatim}
\begin{alltt}
\hlkwd{table}\hlstd{(dbp}\hlopt{$}\hlstd{TRT)}
\end{alltt}
\begin{verbatim}
## 
##  A  B 
## 20 20
\end{verbatim}
\end{kframe}
\end{knitrout}
\end{frame}


\begin{frame}[t,containsverbatim]
\frametitle{Baseline Information}
\begin{knitrout}\footnotesize
\definecolor{shadecolor}{rgb}{0.969, 0.969, 0.969}\color{fgcolor}\begin{kframe}
\begin{alltt}
\hlkwd{par}\hlstd{(}\hlkwc{mfrow}\hlstd{=}\hlkwd{c}\hlstd{(}\hlnum{1}\hlstd{,}\hlnum{2}\hlstd{))}
\hlkwd{boxplot}\hlstd{(DBP1} \hlopt{~} \hlstd{TRT,} \hlkwc{data}\hlstd{=dbp)}
\hlkwd{barplot}\hlstd{(}\hlkwd{table}\hlstd{(dbp}\hlopt{$}\hlstd{Sex, dbp}\hlopt{$}\hlstd{TRT))}
\end{alltt}
\end{kframe}

{\centering \includegraphics[width=0.60\textwidth]{figure/beamer-unnamed-chunk-2-1} 

}



\end{knitrout}
\end{frame}

\begin{frame}[t,containsverbatim]
\frametitle{Pairwise plot}
\begin{knitrout}\footnotesize
\definecolor{shadecolor}{rgb}{0.969, 0.969, 0.969}\color{fgcolor}\begin{kframe}
\begin{alltt}
\hlkwd{pairs}\hlstd{(dbp)}
\end{alltt}
\end{kframe}

{\centering \includegraphics[width=0.60\textwidth]{figure/beamer-unnamed-chunk-3-1} 

}



\end{knitrout}
\end{frame}


\section{Parametric tests}

\begin{frame}[t]
\frametitle{Parametric tests}
\begin{itemize}
	\item 2-groups: t-test for continuous outcome in completely randomized 
		parallel design.
	\item 2-groups: Paired t-test for continuous outcome in crossover design.
	\item 3+-groups: One-way ANOVA for continuous outcome in completely 
		randomized parallel design.
	\item 2+-groups: Two-way ANOVA for continuous outcome in factorial design.
	\item 3+-groups: One-way repeated-measures ANOVA for continuous outcome 
		in randomized block design.
	\item 2-groups: Chisquare test or Fisher's exact test for binary outcome in 
		completely randomized parallel design.
	\item 2-groups: McNemar's test for binary outcome in crossover design.
	\item 3+-groups: Cochrane's Q-test for binary outcome in crossover design.
\end{itemize}
\end{frame}

\subsection{t-test}

\begin{frame}[t, containsverbatim]
\frametitle{Student's t-test for parallel design}
\framesubtitle{Comparing two treament group means with equal variances}
\begin{itemize}
	\item \textbf{Assumption}: $Y_1$ and $Y_2$ are independent 
		and normally distributed with common variance 
		$\sigma^2$.
	\item \textbf{Design}: Randomized parallel design
	\item \textbf{Hypothesis}: $H_0: \mu_1 = \mu_2$ vs. $H_1: \mu_1 \neq \mu_2$
	\item \textbf{Process}
	\begin{itemize}	
		\item Compute the test statistic: 
		$$t = \frac{\bar{y}_1 - \bar{y}_2}{s\sqrt{1/n_1 + 1/n_2}}$$, 
		where
		$$
		\bar{y}_i = \sum_{j=1}^{n_i} y_{ij}/n_i, s = \sqrt{\frac{(n_1-1)s_1^2 + (n_2-1)s_2^2}{n_1+n_2-2}}, 
		s_i^2 = \sum_{j=1}^{n_i} (y_{ij} - \bar{y}_i)^2/(n_i - 1)$$.
		\item Reject $H_0$ if $t > t_{\alpha/2,n_1+n_2-2}$
	\end{itemize}
\end{itemize}
\begin{lstlisting}
t.test(..., var.equal=TRUE)
\end{lstlisting}
\end{frame}

\begin{frame}[t,containsverbatim]
\frametitle{$t$-test with equal variances}
\begin{knitrout}\footnotesize
\definecolor{shadecolor}{rgb}{0.969, 0.969, 0.969}\color{fgcolor}\begin{kframe}
\begin{alltt}
\hlkwd{t.test}\hlstd{(diff} \hlopt{~} \hlstd{TRT,} \hlkwc{data}\hlstd{=dbp,} \hlkwc{var.equal}\hlstd{=}\hlnum{TRUE}\hlstd{)}
\end{alltt}
\begin{verbatim}
## 
## 	Two Sample t-test
## 
## data:  diff by TRT
## t = -12.15, df = 38, p-value = 1.169e-14
## alternative hypothesis: true difference in means is not equal to 0
## 95 percent confidence interval:
##  -12.132758  -8.667242
## sample estimates:
## mean in group A mean in group B 
##           -15.2            -4.8
\end{verbatim}
\end{kframe}
\end{knitrout}
\end{frame}


\begin{frame}[t,containsverbatim]
\frametitle{Welch's t-test for parallel design}
\framesubtitle{Comparing two treatment group means with unequal variances}
\begin{enumerate}
	\item Compute the $t$ statistic
	$$
T = \frac{\bar{y}_1 - \bar{y}_2}{\sqrt{s_1^2/n_1 + s_2^2/n_2}}
	$$
	\item The degree of freedom
	$$
\nu = \left[ \frac{c}{n_1-1} + \frac{(1-c)^2}{n_2-1}\right]^{-1}
	$$
	where
	$$
c = \frac{s_1^2/n_1}{s_1^2/n_1 + s_2^2/n_2}
	$$
	\item Reject $H_0$ if $|T| > t_{\alpha/2, \nu}$
\end{enumerate}
\begin{lstlisting}
t.test(..., var.equal=FALSE)
\end{lstlisting}
\end{frame}

\begin{frame}[t,containsverbatim]
\frametitle{$t$-test with unequal variances}
\begin{knitrout}\footnotesize
\definecolor{shadecolor}{rgb}{0.969, 0.969, 0.969}\color{fgcolor}\begin{kframe}
\begin{alltt}
\hlkwd{t.test}\hlstd{(diff} \hlopt{~} \hlstd{TRT,} \hlkwc{data}\hlstd{=dbp,} \hlkwc{var.equal}\hlstd{=}\hlnum{FALSE}\hlstd{)}
\end{alltt}
\begin{verbatim}
## 
## 	Welch Two Sample t-test
## 
## data:  diff by TRT
## t = -12.15, df = 36.522, p-value = 2.149e-14
## alternative hypothesis: true difference in means is not equal to 0
## 95 percent confidence interval:
##  -12.135063  -8.664937
## sample estimates:
## mean in group A mean in group B 
##           -15.2            -4.8
\end{verbatim}
\end{kframe}
\end{knitrout}
\end{frame}


\begin{frame}[t,containsverbatim]
\frametitle{Wait...Are the two variances equal?}
\begin{knitrout}\footnotesize
\definecolor{shadecolor}{rgb}{0.969, 0.969, 0.969}\color{fgcolor}\begin{kframe}
\begin{alltt}
\hlkwd{var.test}\hlstd{(diff} \hlopt{~} \hlstd{TRT,} \hlkwc{data}\hlstd{=dbp)}
\end{alltt}
\begin{verbatim}
## 
## 	F test to compare two variances
## 
## data:  diff by TRT
## F = 1.5036, num df = 19, denom df = 19, p-value = 0.3819
## alternative hypothesis: true ratio of variances is not equal to 1
## 95 percent confidence interval:
##  0.595142 3.798764
## sample estimates:
## ratio of variances 
##           1.503597
\end{verbatim}
\end{kframe}
\end{knitrout}
\end{frame}


\begin{frame}[t, containsverbatim]
\frametitle{One-sided t-test}
Since ``B'' is a placebo, the one-sided t-test may be more appropriate 
to test the treatment effect:
\begin{knitrout}\footnotesize
\definecolor{shadecolor}{rgb}{0.969, 0.969, 0.969}\color{fgcolor}\begin{kframe}
\begin{alltt}
\hlcom{# data from treatment A and B}
\hlstd{diff.A} \hlkwb{<-} \hlstd{dbp}\hlopt{$}\hlstd{diff[dbp}\hlopt{$}\hlstd{TRT}\hlopt{==}\hlstr{'A'}\hlstd{]}
\hlstd{diff.B} \hlkwb{<-} \hlstd{dbp}\hlopt{$}\hlstd{diff[dbp}\hlopt{$}\hlstd{TRT}\hlopt{==}\hlstr{'B'}\hlstd{]}
\hlcom{# call t.test for one-sided test}
\hlkwd{t.test}\hlstd{(diff.A, diff.B,} \hlkwc{alternative}\hlstd{=}\hlstr{"less"}\hlstd{)}
\end{alltt}
\begin{verbatim}
## 
## 	Welch Two Sample t-test
## 
## data:  diff.A and diff.B
## t = -12.15, df = 36.522, p-value = 1.074e-14
## alternative hypothesis: true difference in means is less than 0
## 95 percent confidence interval:
##       -Inf -8.955466
## sample estimates:
## mean of x mean of y 
##     -15.2      -4.8
\end{verbatim}
\end{kframe}
\end{knitrout}

\end{frame}

\subsection{ANOVA}

\begin{frame}[t]
\frametitle{One-way ANOVA}
The single factor $A$ has $k$ levels: $A_1, A_2, \dots, A_r$, and $k$ patients are allocated to 
each treatment group. We can obtain the samples:
$$
y_{i1}, \dots, y_{in}, i=1, \dots, k
$$
\uncover<2->{\begin{block}{\center Fundamental statistics}
\footnotesize
\begin{itemize}
	\item<3-> Grand mean: $\bar{y}_{*} = \frac{1}{kn}\sum_{i=1}^k \sum_{j=1}^n y_{ij}$
	\item<3-> Marginal mean: $\bar{y}_i = \frac{1}{n}\sum_{j=1}^n y_{ij}$;
	\item<4-> Total sum of squares (SST): $\mathrm{SS}_T = \sum_{i=1}^k \sum_{j=1}^n (y_{ij} - \bar{y}_*)^2$;
	\item<4-> Between-group sum of squares (SSB): $\mathrm{SS}_B = \sum_{i=1}^k (\bar{y}_i - \bar{y}_*)^2$;
	\item<4-> Resitual sum of squares (SSE): $\mathrm{SS}_E = \sum_{i=1}^k \sum_{j=1}^n (y_{ij} - \bar{y}_i)^2$;	
\end{itemize}
\end{block}
}
\end{frame}


\begin{frame}[t]
\frametitle{One-way ANOVA Table}
\begin{table}
\caption{One-way ANOVA Table with $k$ groups and $n$ subjects}
\begin{tabular}{lcccc}
\hline
Variance & $\textrm{SS}$ & $\textrm{DF}$ & $\textrm{MS}$ & $F$-value\\
\hline
Between-groups & $\textrm{SS}_b$ & $k-1$ & $\textrm{MS}_b$ & $F = \textrm{MS}_b/\textrm{MS}_w$\\
Within-groups & $\textrm{SS}_w$ & $n-k$ & $\textrm{MS}_w$ & \\
Total & $\textrm{SS}_T$ & $n-1$ & $\textrm{MS}_T$ & \\
\hline
\end{tabular}
\end{table}
\end{frame}

\begin{frame}[t,containsverbatim]
\frametitle{One-way ANOVA for Time Changes}
Since the treatment period in the DBP trial was measured at months 1, 
2, 3 and 4 post baseline. To see the mean changes over the periods:
\begin{knitrout}\footnotesize
\definecolor{shadecolor}{rgb}{0.969, 0.969, 0.969}\color{fgcolor}\begin{kframe}
\begin{alltt}
\hlkwd{aggregate}\hlstd{(dbp[,}\hlnum{3}\hlopt{:}\hlnum{7}\hlstd{],} \hlkwd{list}\hlstd{(}\hlkwc{TRT}\hlstd{=dbp}\hlopt{$}\hlstd{TRT), mean)}
\end{alltt}
\begin{verbatim}
##   TRT   DBP1  DBP2   DBP3   DBP4   DBP5
## 1   A 116.55 113.5 110.70 106.25 101.35
## 2   B 116.75 115.2 114.05 112.45 111.95
\end{verbatim}
\end{kframe}
\end{knitrout}
Now we can employ the one-way ANOVA to test the change over time. But the first thing is to ``reshape'' the data:
\begin{knitrout}\footnotesize
\definecolor{shadecolor}{rgb}{0.969, 0.969, 0.969}\color{fgcolor}\begin{kframe}
\begin{alltt}
\hlstd{Dbp} \hlkwb{<-} \hlkwd{reshape}\hlstd{(dbp,} \hlkwc{direction}\hlstd{=}\hlstr{"long"}\hlstd{,}
        \hlkwc{varying}\hlstd{=}\hlkwd{paste}\hlstd{(}\hlstr{"DBP"}\hlstd{,} \hlnum{1}\hlopt{:}\hlnum{5}\hlstd{,} \hlkwc{sep}\hlstd{=}\hlstr{""}\hlstd{),}
        \hlkwc{idvar} \hlstd{=} \hlkwd{c}\hlstd{(}\hlstr{"Subject"}\hlstd{,} \hlstr{"TRT"}\hlstd{,} \hlstr{"Age"}\hlstd{,} \hlstr{"Sex"}\hlstd{,} \hlstr{"diff"}\hlstd{),} \hlkwc{sep}\hlstd{=}\hlstr{""}\hlstd{)}
\hlkwd{colnames}\hlstd{(Dbp)} \hlkwb{<-} \hlkwd{c}\hlstd{(}\hlstr{"Subject"}\hlstd{,} \hlstr{"TRT"}\hlstd{,} \hlstr{"Age"}\hlstd{,} \hlstr{"Sex"}\hlstd{,} \hlstr{"diff"}\hlstd{,} \hlstr{"Time"}\hlstd{,} \hlstr{"DBP"}\hlstd{)}
\hlkwd{rownames}\hlstd{(Dbp)} \hlkwb{<-} \hlkwa{NULL}
\hlstd{Dbp}\hlopt{$}\hlstd{Time} \hlkwb{<-} \hlkwd{as.factor}\hlstd{(Dbp}\hlopt{$}\hlstd{Time)}
\hlkwd{head}\hlstd{(Dbp,} \hlnum{3}\hlstd{)}
\end{alltt}
\begin{verbatim}
##   Subject TRT Age Sex diff Time DBP
## 1       1   A  43   F   -9    1 114
## 2       2   A  51   M  -15    1 116
## 3       3   A  48   F  -21    1 119
\end{verbatim}
\end{kframe}
\end{knitrout}
\end{frame}

\begin{frame}[t,containsverbatim]
\frametitle{One-way ANOVA for Two Treatments}
\begin{knitrout}\footnotesize
\definecolor{shadecolor}{rgb}{0.969, 0.969, 0.969}\color{fgcolor}\begin{kframe}
\begin{alltt}
\hlcom{# test for treatment A}
\hlstd{dbpA} \hlkwb{<-} \hlstd{Dbp[Dbp}\hlopt{$}\hlstd{TRT}\hlopt{==}\hlstr{'A'}\hlstd{,]}
\hlstd{test.A} \hlkwb{<-} \hlkwd{aov}\hlstd{(DBP} \hlopt{~} \hlstd{Time, dbpA)}
\hlkwd{summary}\hlstd{(test.A)}
\end{alltt}
\begin{verbatim}
##             Df Sum Sq Mean Sq F value Pr(>F)    
## Time         4 2879.7   719.9     127 <2e-16 ***
## Residuals   95  538.5     5.7                   
## ---
## Signif. codes:  0 '***' 0.001 '**' 0.01 '*' 0.05 '.' 0.1 ' ' 1
\end{verbatim}
\end{kframe}
\end{knitrout}

\begin{knitrout}\footnotesize
\definecolor{shadecolor}{rgb}{0.969, 0.969, 0.969}\color{fgcolor}\begin{kframe}
\begin{alltt}
\hlcom{# test for treatment B}
\hlstd{dbpB} \hlkwb{<-} \hlstd{Dbp[Dbp}\hlopt{$}\hlstd{TRT}\hlopt{==}\hlstr{'B'}\hlstd{, ]}
\hlstd{test.B} \hlkwb{<-} \hlkwd{aov}\hlstd{(DBP} \hlopt{~} \hlstd{Time, dbpB)}
\hlkwd{summary}\hlstd{(test.B)}
\end{alltt}
\begin{verbatim}
##             Df Sum Sq Mean Sq F value  Pr(>F)    
## Time         4  311.6   77.89   17.63 7.5e-11 ***
## Residuals   95  419.8    4.42                    
## ---
## Signif. codes:  0 '***' 0.001 '**' 0.01 '*' 0.05 '.' 0.1 ' ' 1
\end{verbatim}
\end{kframe}
\end{knitrout}
\end{frame}


\begin{frame}[t,containsverbatim]
\frametitle{Post-hoc Tests for ANOVA of treatment A}
\begin{knitrout}\footnotesize
\definecolor{shadecolor}{rgb}{0.969, 0.969, 0.969}\color{fgcolor}\begin{kframe}
\begin{alltt}
\hlkwd{TukeyHSD}\hlstd{(test.A)}
\end{alltt}
\begin{verbatim}
##   Tukey multiple comparisons of means
##     95% family-wise confidence level
## 
## Fit: aov(formula = DBP ~ Time, data = dbpA)
## 
## $Time
##       diff        lwr         upr     p adj
## 2-1  -3.05  -5.143586  -0.9564144 0.0009687
## 3-1  -5.85  -7.943586  -3.7564144 0.0000000
## 4-1 -10.30 -12.393586  -8.2064144 0.0000000
## 5-1 -15.20 -17.293586 -13.1064144 0.0000000
## 3-2  -2.80  -4.893586  -0.7064144 0.0030529
## 4-2  -7.25  -9.343586  -5.1564144 0.0000000
## 5-2 -12.15 -14.243586 -10.0564144 0.0000000
## 4-3  -4.45  -6.543586  -2.3564144 0.0000005
## 5-3  -9.35 -11.443586  -7.2564144 0.0000000
## 5-4  -4.90  -6.993586  -2.8064144 0.0000000
\end{verbatim}
\end{kframe}
\end{knitrout}
\end{frame}


\begin{frame}[t,containsverbatim]
\frametitle{Post-hoc Tests for ANOVA of treatment B}
\begin{knitrout}\footnotesize
\definecolor{shadecolor}{rgb}{0.969, 0.969, 0.969}\color{fgcolor}\begin{kframe}
\begin{alltt}
\hlkwd{TukeyHSD}\hlstd{(test.B)}
\end{alltt}
\begin{verbatim}
##   Tukey multiple comparisons of means
##     95% family-wise confidence level
## 
## Fit: aov(formula = DBP ~ Time, data = dbpB)
## 
## $Time
##      diff       lwr        upr     p adj
## 2-1 -1.55 -3.398584  0.2985843 0.1440046
## 3-1 -2.70 -4.548584 -0.8514157 0.0009333
## 4-1 -4.30 -6.148584 -2.4514157 0.0000000
## 5-1 -4.80 -6.648584 -2.9514157 0.0000000
## 3-2 -1.15 -2.998584  0.6985843 0.4207789
## 4-2 -2.75 -4.598584 -0.9014157 0.0007122
## 5-2 -3.25 -5.098584 -1.4014157 0.0000400
## 4-3 -1.60 -3.448584  0.2485843 0.1223788
## 5-3 -2.10 -3.948584 -0.2514157 0.0176793
## 5-4 -0.50 -2.348584  1.3485843 0.9433857
\end{verbatim}
\end{kframe}
\end{knitrout}
\end{frame}


\begin{frame}[t,containsverbatim]
\frametitle{Two-way ANOVA With Interaction}
The DBP trial has two factors: treatment and Time. Under this 
situation, one-way ANOVA (within treatment groups across Time) 
cannot capture the interaction between these two factors.

Therefore, a two-way or multi-way ANOVA is needed to analyze the 
interaction before making statistical inferences about the main 
effects.

\begin{knitrout}\footnotesize
\definecolor{shadecolor}{rgb}{0.969, 0.969, 0.969}\color{fgcolor}\begin{kframe}
\begin{alltt}
\hlstd{mod} \hlkwb{<-} \hlkwd{aov}\hlstd{(DBP} \hlopt{~} \hlstd{TRT}\hlopt{*}\hlstd{Time, Dbp)}
\hlkwd{summary}\hlstd{(mod)}
\end{alltt}
\begin{verbatim}
##              Df Sum Sq Mean Sq F value Pr(>F)    
## TRT           1  972.4   972.4  192.81 <2e-16 ***
## Time          4 2514.1   628.5  124.62 <2e-16 ***
## TRT:Time      4  677.1   169.3   33.56 <2e-16 ***
## Residuals   190  958.3     5.0                   
## ---
## Signif. codes:  0 '***' 0.001 '**' 0.01 '*' 0.05 '.' 0.1 ' ' 1
\end{verbatim}
\end{kframe}
\end{knitrout}
\end{frame}


\begin{frame}[t,containsverbatim]
\frametitle{Plot Interaction Between Time and Treatment}
\begin{knitrout}\footnotesize
\definecolor{shadecolor}{rgb}{0.969, 0.969, 0.969}\color{fgcolor}\begin{kframe}
\begin{alltt}
\hlkwd{par}\hlstd{(}\hlkwc{mfrow}\hlstd{=}\hlkwd{c}\hlstd{(}\hlnum{2}\hlstd{,}\hlnum{1}\hlstd{),} \hlkwc{mar}\hlstd{=}\hlkwd{c}\hlstd{(}\hlnum{5}\hlstd{,}\hlnum{3}\hlstd{,}\hlnum{1}\hlstd{,}\hlnum{1}\hlstd{))}
\hlkwd{with}\hlstd{(Dbp,} \hlkwd{interaction.plot}\hlstd{(Time,TRT,DBP,}\hlkwc{las}\hlstd{=}\hlnum{1}\hlstd{,}\hlkwc{legend}\hlstd{=T))}
\hlkwd{with}\hlstd{(Dbp,} \hlkwd{interaction.plot}\hlstd{(TRT,Time,DBP,}\hlkwc{las}\hlstd{=}\hlnum{1}\hlstd{,}\hlkwc{legend}\hlstd{=T))}
\end{alltt}
\end{kframe}

{\centering \includegraphics[width=0.60\textwidth]{figure/beamer-unnamed-chunk-15-1} 

}



\end{knitrout}
\end{frame}


\begin{frame}[t,containsverbatim]
\frametitle{Post-Hoc Analysis}
\begin{knitrout}\footnotesize
\definecolor{shadecolor}{rgb}{0.969, 0.969, 0.969}\color{fgcolor}\begin{kframe}
\begin{alltt}
\hlkwd{TukeyHSD}\hlstd{(}\hlkwd{aov}\hlstd{(DBP} \hlopt{~} \hlstd{TRT}\hlopt{*}\hlstd{Time, Dbp))}
\end{alltt}
\begin{verbatim}
##   Tukey multiple comparisons of means
##     95% family-wise confidence level
## 
## Fit: aov(formula = DBP ~ TRT * Time, data = Dbp)
## 
## $TRT
##     diff      lwr      upr p adj
## B-A 4.41 3.783529 5.036471     0
## 
## $Time
##        diff        lwr        upr     p adj
## 2-1  -2.300  -3.683042 -0.9169576 0.0000816
## 3-1  -4.275  -5.658042 -2.8919576 0.0000000
## 4-1  -7.300  -8.683042 -5.9169576 0.0000000
## 5-1 -10.000 -11.383042 -8.6169576 0.0000000
## 3-2  -1.975  -3.358042 -0.5919576 0.0011017
## 4-2  -5.000  -6.383042 -3.6169576 0.0000000
## 5-2  -7.700  -9.083042 -6.3169576 0.0000000
## 4-3  -3.025  -4.408042 -1.6419576 0.0000001
## 5-3  -5.725  -7.108042 -4.3419576 0.0000000
## 5-4  -2.700  -4.083042 -1.3169576 0.0000022
## 
## $`TRT:Time`
##           diff         lwr         upr     p adj
## B:1-A:1   0.20  -2.0737649   2.4737649 0.9999998
## A:2-A:1  -3.05  -5.3237649  -0.7762351 0.0011351
## B:2-A:1  -1.35  -3.6237649   0.9237649 0.6684315
## A:3-A:1  -5.85  -8.1237649  -3.5762351 0.0000000
## B:3-A:1  -2.50  -4.7737649  -0.2262351 0.0188264
## A:4-A:1 -10.30 -12.5737649  -8.0262351 0.0000000
## B:4-A:1  -4.10  -6.3737649  -1.8262351 0.0000014
## A:5-A:1 -15.20 -17.4737649 -12.9262351 0.0000000
## B:5-A:1  -4.60  -6.8737649  -2.3262351 0.0000000
## A:2-B:1  -3.25  -5.5237649  -0.9762351 0.0003579
## B:2-B:1  -1.55  -3.8237649   0.7237649 0.4723958
## A:3-B:1  -6.05  -8.3237649  -3.7762351 0.0000000
## B:3-B:1  -2.70  -4.9737649  -0.4262351 0.0072480
## A:4-B:1 -10.50 -12.7737649  -8.2262351 0.0000000
## B:4-B:1  -4.30  -6.5737649  -2.0262351 0.0000003
## A:5-B:1 -15.40 -17.6737649 -13.1262351 0.0000000
## B:5-B:1  -4.80  -7.0737649  -2.5262351 0.0000000
## B:2-A:2   1.70  -0.5737649   3.9737649 0.3355035
## A:3-A:2  -2.80  -5.0737649  -0.5262351 0.0043660
## B:3-A:2   0.55  -1.7237649   2.8237649 0.9988534
## A:4-A:2  -7.25  -9.5237649  -4.9762351 0.0000000
## B:4-A:2  -1.05  -3.3237649   1.2237649 0.8990806
## A:5-A:2 -12.15 -14.4237649  -9.8762351 0.0000000
## B:5-A:2  -1.55  -3.8237649   0.7237649 0.4723958
## A:3-B:2  -4.50  -6.7737649  -2.2262351 0.0000001
## B:3-B:2  -1.15  -3.4237649   1.1237649 0.8372192
## A:4-B:2  -8.95 -11.2237649  -6.6762351 0.0000000
## B:4-B:2  -2.75  -5.0237649  -0.4762351 0.0056388
## A:5-B:2 -13.85 -16.1237649 -11.5762351 0.0000000
## B:5-B:2  -3.25  -5.5237649  -0.9762351 0.0003579
## B:3-A:3   3.35   1.0762351   5.6237649 0.0001963
## A:4-A:3  -4.45  -6.7237649  -2.1762351 0.0000001
## B:4-A:3   1.75  -0.5237649   4.0237649 0.2948066
## A:5-A:3  -9.35 -11.6237649  -7.0762351 0.0000000
## B:5-A:3   1.25  -1.0237649   3.5237649 0.7590918
## A:4-B:3  -7.80 -10.0737649  -5.5262351 0.0000000
## B:4-B:3  -1.60  -3.8737649   0.6737649 0.4247400
## A:5-B:3 -12.70 -14.9737649 -10.4262351 0.0000000
## B:5-B:3  -2.10  -4.3737649   0.1737649 0.0975920
## B:4-A:4   6.20   3.9262351   8.4737649 0.0000000
## A:5-A:4  -4.90  -7.1737649  -2.6262351 0.0000000
## B:5-A:4   5.70   3.4262351   7.9737649 0.0000000
## A:5-B:4 -11.10 -13.3737649  -8.8262351 0.0000000
## B:5-B:4  -0.50  -2.7737649   1.7737649 0.9994658
## B:5-A:5  10.60   8.3262351  12.8737649 0.0000000
\end{verbatim}
\end{kframe}
\end{knitrout}
\end{frame}

\subsection{ANCOVA}

\begin{frame}[t]
\frametitle{What is Analysis of Covariance (ANCOVA)?}
\begin{itemize}[<+->]
	\item One-way/two-way/Multi-way ANOVA with uncontrollable but measurable 
		independent variable $X$;
	\item The outcome $Y$ is continuous;
	\item $Y$ and $X$ (covariate) has linear relationship;
	\item Regression analysis is used to adjust for the effect of $X$ on $Y$.
	\item Here we does not control for $X$ in the trials, but conduct a post-hoc 
		analysis.
\end{itemize}
\uncover<6->{
\begin{alertblock}{\center Example}
\begin{itemize}
	\item Initial blood pressure ($X$) for the BP reductions ($Y$) in comparing different 
		blood pressure medications ($G$).
	\item Parallel-group clinical trials with pre-treatment baselines.  
\end{itemize}
\end{alertblock}
}
\end{frame}


\begin{frame}[t]
\frametitle{One-way ANCOVA}
The single factor $A$ has $r$ levels: $A_1, A_2, \dots, A_r$, and $s$ patients are allocated to 
each treatment group. We can obtain the samples:
$$
(x_{i1}, y_{i1}), \dots, (x_{is}, y_{is}), i=1, \dots, r
$$
\uncover<2->{\begin{block}{\center Fundamental statistics}
\footnotesize
\begin{itemize}
	\item<3-> $\bar{x}_{i*} = \frac{1}{s}\sum_{j=1}^s x_{ij}; \bar{x}_{**} = \frac{1}{rs}\sum_{i=1}^r \sum_{j=1}^s x_{ij}$
	\item<3-> $\bar{y}_{i*} = \frac{1}{s}\sum_{j=1}^s y_{ij}; \bar{y}_{**} = \frac{1}{rs}\sum_{i=1}^r \sum_{j=1}^s y_{ij}$
	\item<4-> $\mathrm{SST}(x) = \sum_{i=1}^r \sum_{j=1}^s (x_{ij} - x_{**})^2$;
	\item<4-> $\mathrm{SSA}(x) = \sum_{i=1}^r (x_{i*} - x_{**})^2$;
	\item<4-> $\mathrm{SSE}(x) = \sum_{i=1}^r \sum_{j=1}^s (x_{ij} - x_{i*})^2$;
	\item<5-> $\mathrm{SST}(y) = \sum_{i=1}^r \sum_{j=1}^s (y_{ij} - y_{**})^2$;
	\item<5-> $\mathrm{SSA}(y) = \sum_{i=1}^r (y_{i*} - y_{**})^2$;
	\item<5-> $\mathrm{SSE}(y) = \sum_{i=1}^r \sum_{j=1}^s (y_{ij} - y_{i*})^2$;
	\item<6-> $\mathrm{SPT} = \sum_{i=1}^r \sum_{j=1}^s (x_{ij}-x_{**})(y_{ij}-y_{**})^2$;
	\item<6-> $\mathrm{SPA} = \sum_{i=1}^r (x_{i*} - x_{**})(y_{i*} - y_{**})$;
	\item<6-> $\mathrm{SPE} = \sum_{i=1}^r \sum_{j=1}^s (x_{ij}-x_{i*})(y_{ij} - y_{i*})$.	
\end{itemize}
\end{block}
}
\end{frame}


\begin{frame}[t]
\frametitle{One-way ANCOVA: Procedure}
\begin{enumerate}[<+->]
	\item Compute the above statistics and group them into a table:
\begin{table}
\footnotesize
\begin{tabular}{l|c|c|c|c}
\hline
Variance & $\textrm{SS}(x)$ & $\textrm{SS}(y)$ & $\textrm{SP}$ & $\textrm{DF}$\\
\hline
inter-group & $\textrm{SSA}(x)$ & $\textrm{SSA}(y)$ & $\textrm{SPA}$ & $r-1$\\
intra-group & $\textrm{SSE}(x)$ & $\textrm{SSE}(y)$ & $\textrm{SPE}$ & $r(s-1)$\\
Total & $\textrm{SST}(x)$ & $\textrm{SST}(y)$ & $\textrm{SPT}$ & $rs-1$\\
\hline
\end{tabular}
\end{table}
	\item Compute the intra-group regresson coefficient $\beta = \frac{\textrm{SPE}}{\textrm{SSE}(x)}$;
	\item If significant, adjust for the linear regression:
	$$
\bar{y}_{i*}(x = \bar{x}_{**}) = \bar{y}_{i*} - \beta(\bar{x}_{i*} - \bar{x}_{**})
	$$
	\item $Q_E = \textrm{SSE}(y) - \frac{(\textrm{SPE})^2}{\textrm{SSE}(x)}$;
	\item $Q_T = \textrm{SST}(y) - \frac{(\textrm{SPT})^2}{\textrm{SST}(x)}$;
	\item $Q_A = Q_T - Q_E$;
	\item $MQ_A = \frac{Q_A}{r-1}$; $MQ_E = \frac{Q_E}{r(s-1)}$
	\item $F=\frac{MQ_A}{MQ_E} \sim F_{r-1, r(s-1)-1}$
\end{enumerate}
\end{frame}


\begin{frame}[t]
\frametitle{One-way ANCOVA: Procedure}
\uncover<1->{\begin{table}
\footnotesize
\begin{tabular}{l|c|c|c|c}
\hline
Variance & $\textrm{SS}(x)$ & $\textrm{SS}(y)$ & $\textrm{SP}$ & $\textrm{DF}$\\
\hline
inter-group & $\textrm{SSA}(x)$ & $\textrm{SSA}(y)$ & $\textrm{SPA}$ & $r-1$\\
intra-group & $\textrm{SSE}(x)$ & $\textrm{SSE}(y)$ & $\textrm{SPE}$ & $r(s-1)$\\
Total & $\textrm{SST}(x)$ & $\textrm{SST}(y)$ & $\textrm{SPT}$ & $rs-1$\\
\hline
\end{tabular}
\end{table}}

\uncover<2->{
\alert{\begin{center} $\Downarrow$ \end{center}}
}

\uncover<3->{\begin{table}
\footnotesize
\caption{Adjusted ANCOVA Table}
\begin{tabular}{l|c|c|c|c|c}
\hline
Variance & $\textrm{SS}$ & $\textrm{DF}$ & $\textrm{MSS}$ & $\textrm{F}$ & Significance\\
\hline
inter-group & $Q_A$ & $r-1$ & $MQ_A$ & $F$ & \\
intra-group & $Q_E$ & $r(s-1)-1$ & $MQ_E$ & & \\
Total & $Q_T$ & $rs-2$ & & & \\
\hline
\end{tabular}
\end{table}}

\uncover<4->{\begin{alertblock}{\center Conclusion}

\end{alertblock}}
\end{frame}


\begin{frame}[t]
\frametitle{Two-way ANCOVA: No Interaction}
\begin{itemize}[<+->]
	\item Two factors $A: A_1, \dots, A_r$ and $B: B_1, \dots, B_s$;
	\item Continuous covariate $X$;
	\item Continuous outcome $Y$;
\end{itemize}
\uncover<4->{\begin{table}
\footnotesize
\caption{Adjusted ANCOVA Table Without Interaction}
\begin{tabular}{l|c|c|c|c|c}
\hline
Variance & $\textrm{SS}$ & $\textrm{DF}$ & $\textrm{MSS}$ & $\textrm{F}$ & Significance\\
\hline
inter-A & $Q_A$ & $r-1$ & $MQ_A$ & $F_A$ & \\
inter-B & $Q_B$ & $s-1$ & $MQ_B$ & $F_B$ & \\
intra-group & $Q_E$ & $(r-1)(s-1)-1$ & $MQ_E$ & & \\
Total & $Q_T$ & $rs-2$ & & & \\
\hline
\end{tabular}
\end{table}}
\end{frame}

\begin{frame}[t]
\frametitle{Two-way ANCOVA: With Interaction}
\begin{itemize}[<+->]
	\item Two factors $A: A_1, \dots, A_r$ and $B: B_1, \dots, B_s$;
	\item Continuous covariate $X$;
	\item Continuous outcome $Y$;
	\item Each cell with $m$ observations.
\end{itemize}
\uncover<5->{\begin{table}
\footnotesize
\caption{Adjusted ANCOVA Table With Interaction}
\begin{tabular}{l|c|c|c|c|c}
\hline
Variance & $\textrm{SS}$ & $\textrm{DF}$ & $\textrm{MSS}$ & $\textrm{F}$ & Significance\\
\hline
inter-A & $Q_A$ & $r-1$ & $MQ_A$ & $F_A$ & \\
inter-B & $Q_B$ & $s-1$ & $MQ_B$ & $F_B$ & \\
inter-AB & $Q_{AB}$ & $(r-1)(s-1)$ & $MQ_{AB}$ & $F_{AB}$ & \\
intra-group & $Q_E$ & $rs(m-1)-1$ & $MQ_E$ & & \\
Total & $Q_T$ & $rsm-2$ & & & \\
\hline
\end{tabular}
\end{table}}
\end{frame}


\begin{frame}[t,containsverbatim]
\frametitle{Analysis of DBP Change from Baseline with ANCOVA}
\begin{itemize}
	\item We now analyze the change from baseline in DBP at the 
		end of trial which is defined as ``diff''.
	\item We start from the full model containing all 
		``covariates''.
	\item Perform backward stepwise model selection to simplify 
		the model:  
\end{itemize}
\begin{knitrout}\footnotesize
\definecolor{shadecolor}{rgb}{0.969, 0.969, 0.969}\color{fgcolor}\begin{kframe}
\begin{alltt}
\hlcom{# start with the full model}
\hlstd{m0} \hlkwb{<-} \hlkwd{lm}\hlstd{(diff} \hlopt{~} \hlstd{TRT}\hlopt{*}\hlstd{Age}\hlopt{*}\hlstd{Sex, dbp)}
\hlcom{# stepwise model selection}
\hlstd{m1} \hlkwb{=} \hlkwd{step}\hlstd{(m0)}
\end{alltt}
\begin{verbatim}
## Start:  AIC=79.52
## diff ~ TRT * Age * Sex
## 
##               Df Sum of Sq    RSS   AIC
## - TRT:Age:Sex  1    2.7059 198.47 78.07
## <none>                     195.76 79.52
## 
## Step:  AIC=78.07
## diff ~ TRT + Age + Sex + TRT:Age + TRT:Sex + Age:Sex
## 
##           Df Sum of Sq    RSS    AIC
## - TRT:Sex  1    1.3256 199.79 76.336
## - TRT:Age  1    9.5638 208.03 77.952
## <none>                 198.47 78.070
## - Age:Sex  1   17.0694 215.53 79.370
## 
## Step:  AIC=76.34
## diff ~ TRT + Age + Sex + TRT:Age + Age:Sex
## 
##           Df Sum of Sq    RSS    AIC
## <none>                 199.79 76.336
## - TRT:Age  1    10.272 210.06 76.341
## - Age:Sex  1    16.164 215.96 77.448
\end{verbatim}
\end{kframe}
\end{knitrout}
\end{frame}


\begin{frame}[t,containsverbatim]
\frametitle{ANCOVA Analysis of the Changes from Baseline}
\begin{knitrout}\footnotesize
\definecolor{shadecolor}{rgb}{0.969, 0.969, 0.969}\color{fgcolor}\begin{kframe}
\begin{alltt}
\hlcom{# fit the reduced model}
\hlstd{m2} \hlkwb{<-} \hlkwd{lm}\hlstd{(diff} \hlopt{~} \hlstd{TRT} \hlopt{+} \hlstd{Age, dbp)}
\hlcom{# output the anova result}
\hlkwd{anova}\hlstd{(m2)}
\end{alltt}
\begin{verbatim}
## Analysis of Variance Table
## 
## Response: diff
##           Df  Sum Sq Mean Sq  F value    Pr(>F)    
## TRT        1 1081.60 1081.60 176.0395 1.228e-15 ***
## Age        1   51.07   51.07   8.3119  0.006525 ** 
## Residuals 37  227.33    6.14                       
## ---
## Signif. codes:  0 '***' 0.001 '**' 0.01 '*' 0.05 '.' 0.1 ' ' 1
\end{verbatim}
\end{kframe}
\end{knitrout}
\end{frame}

\subsection{Repeated Measure ANOVA}

\begin{frame}[t]
\frametitle{What is repeated-measure ANOVA?}
\begin{itemize}
	\item<1-> The equivalent of the one-way ANOVA for correlated/non-independent groups
	\begin{itemize}
		\item Extension of paired $t$-test;
		\item Within-subjects ANOVA;
		\item ANOVA for correlated samples;
	\end{itemize}
	\item<2-> Here we only consider the one-way repeated measure ANOVA:
	\begin{itemize}
		\item One independent categorical variable (nominal or ordinal)
		\item One continuous dependent variable (interval or ratio)
	\end{itemize}
\end{itemize}
\only<2>{\begin{figure}
\includegraphics[width=0.50\textwidth]{images/repanova-time.png}
\end{figure}}
\only<3>{\begin{figure}
\includegraphics[width=0.50\textwidth]{images/repanova-treatment.png}
\end{figure}}
\uncover<4->{\begin{alertblock}{\center When to use repeated-measure ANOVA?}
\begin{itemize}
	\item Changes in mean scores over three or more time points;
	\item Differences in mean scores under three or more different conditions.
\end{itemize}
\end{alertblock}}
\end{frame}


\begin{frame}[t]
\frametitle{Logics of Repeated Measures ANOVA}
\begin{itemize}
	\item<1-> ANOVA partitions total variability ($\textrm{SS}_T$) into between-groups 
		variability ($\textrm{SS}_b$) and within-groups variability ($\textrm{SS}_w$)
	\item<2-> Within-group variability ($\textrm{SS}_w$) is defined as the error variability 
		($\textrm{SS}_{error}$).
	\item<3-> Mean sum of squares for between-groups ($\textrm{MS}_w$) and within-groups 
		($\textrm{MS}_w$)
	\item<4-> \alert{Independent ANOVA: $F = \frac{\textrm{MS}_b}{\textrm{MS}_w} = \frac{\textrm{MS}_b}{\textrm{MS}_{error}}$}
	\item<5-> \alert{Repeated measures ANOVA: $F = \frac{\textrm{MS}_b}{\textrm{MS}_w} = \frac{\textrm{MS}_{condition}}{\textrm{MS}_{error}}$}
	\item<6-> $\textrm{SS}_{error} = \textrm{SS}_w - \text{SS}_{subject} = \textrm{SS}_T - \textrm{SS}_{condition} - \textrm{SS}_{subject}$
	\item<7-> $\textrm{MS}_{error} = \textrm{SS}_{error}/(k-1)(n-1)$, where $k$ is the numer of groups (time points or treatments), and $n$ is 
		the number of subjects.
	\item<8> $\textrm{SS}_{subject} = k \times \sum_{i=1}^n (\bar{x}_i - \bar{x})^2$, where $\bar{x}_i$ is the mean for subject $i$.
\end{itemize}
\only<4>{
\begin{figure}
\includegraphics[width=0.6\textwidth]{images/anova-partition-ss-ind.png}
\end{figure}}
\only<5>{
\begin{figure}
\includegraphics[width=0.6\textwidth]{images/repanova-partition-ss.png}
\end{figure}}
\end{frame}


\begin{frame}[t,containsverbatim]
\frametitle{Repeated-measures ANOVA for DBP Trial}
\begin{knitrout}\footnotesize
\definecolor{shadecolor}{rgb}{0.969, 0.969, 0.969}\color{fgcolor}\begin{kframe}
\begin{alltt}
\hlstd{Dbp2} \hlkwb{<-} \hlkwd{reshape}\hlstd{(dbp,}\hlkwc{direction}\hlstd{=}\hlstr{"long"}\hlstd{,}
  \hlkwc{varying}\hlstd{=}\hlkwd{paste}\hlstd{(}\hlstr{"DBP"}\hlstd{,} \hlnum{2}\hlopt{:}\hlnum{5}\hlstd{,} \hlkwc{sep}\hlstd{=}\hlstr{""}\hlstd{),}
  \hlkwc{idvar} \hlstd{=} \hlkwd{c}\hlstd{(}\hlstr{"Subject"}\hlstd{,} \hlstr{"TRT"}\hlstd{,} \hlstr{"Age"}\hlstd{,} \hlstr{"Sex"}\hlstd{,} \hlstr{"DBP1"}\hlstd{),} \hlkwc{sep}\hlstd{=}\hlstr{""}\hlstd{)}
\hlkwd{rownames}\hlstd{(Dbp2)} \hlkwb{<-} \hlkwa{NULL}
\hlkwd{head}\hlstd{(Dbp2)}
\end{alltt}
\begin{verbatim}
##   Subject TRT DBP1 Age Sex diff time DBP
## 1       1   A  114  43   F   -9    2 115
## 2       2   A  116  51   M  -15    2 113
## 3       3   A  119  48   F  -21    2 115
## 4       4   A  115  42   F  -14    2 113
## 5       5   A  116  49   M  -11    2 112
## 6       6   A  117  47   M  -15    2 112
\end{verbatim}
\end{kframe}
\end{knitrout}
\end{frame}


\begin{frame}[t,containsverbatim]
\frametitle{Repeated-measures ANOVA for DBP Trial}
\begin{knitrout}\footnotesize
\definecolor{shadecolor}{rgb}{0.969, 0.969, 0.969}\color{fgcolor}\begin{kframe}
\begin{alltt}
\hlstd{m2} \hlkwb{<-} \hlkwd{aov}\hlstd{(DBP} \hlopt{~} \hlstd{DBP1} \hlopt{+} \hlstd{TRT} \hlopt{+} \hlkwd{Error}\hlstd{(Subject}\hlopt{/}\hlstd{time),} \hlkwc{data}\hlstd{=Dbp2)}
\hlkwd{summary}\hlstd{(m2)}
\end{alltt}
\begin{verbatim}
## 
## Error: Subject
##      Df Sum Sq Mean Sq
## DBP1  1  978.1   978.1
## 
## Error: Subject:time
##           Df Sum Sq Mean Sq F value Pr(>F)
## Residuals  1  535.7   535.7               
## 
## Error: Within
##            Df Sum Sq Mean Sq F value   Pr(>F)    
## DBP1        1      8    7.99   0.605    0.438    
## TRT         1    252  252.01  19.082 2.28e-05 ***
## Residuals 155   2047   13.21                     
## ---
## Signif. codes:  0 '***' 0.001 '**' 0.01 '*' 0.05 '.' 0.1 ' ' 1
\end{verbatim}
\end{kframe}
\end{knitrout}
\end{frame}


\begin{frame}[t,containsverbatim]
\frametitle{\texttt{aov()} function}
\begin{lstlisting}[language=R]
aov(formula, data=data.frame)
\end{lstlisting}

\begin{table}
\caption{Special Symbols}
\small
\begin{tabular}{p{0.2\textwidth}p{0.70\textwidth}}
Symbol	&	Usage\\
\hline
\verb|~| & The delimiter between response variable and explanatory variables. \\
\verb|+| & The delimiter between explanatory variables.\\
\verb|:| & The interaction between explanatory variables.\\
\verb|*| & All possible interaction terms.\\
\verb|^| & The highest order of interaction.\\
\verb|.| & All the independent variables.\\
\hline
\end{tabular}
\end{table}
\end{frame}


\begin{frame}[t,containsverbatim]
\frametitle{ANOVA Design Settings}
\begin{table}
\caption{ANOVA Design}
\begin{tabular}{p{0.3\textwidth}p{0.60\textwidth}}
\hline
Expression	&	Meaning\\
\hline
\verb#y ~ A# & One-way ANOVA. \\
\verb#y~x+A# & One-way ANCOVA with one covariate $x$.\\
\verb#y~A*B# & Two-way ANOVA for factorial design.\\
\verb#y~x1+x2+A*B# & Two-way ANOVA with two covariates in a factorial design.\\
\verb#y~B+A# & Randomized block design ($B$ is the block factor).\\
\verb#y ~ A + Error(Subject)# & Cross-over design.\\
\hline
\end{tabular}
\end{table}
\end{frame}


\begin{frame}[t,containsverbatim]
\frametitle{Pearson's Chi-square test}
\framesubtitle{Comparing categorical outcomes across different treatment groups}
\begin{itemize}
	\item Setting: $m$ treatments, $n$ categories of outcomes
 	\item Compute the test statistic:
	$$
\chi^2 = \sum_{i} \frac{(O_i - E_i)^2}{E_i} \sim \chi^2(df=(m-1)(n-1))
	$$
	\item Use Yates' correction for lack of continuity When 
		the expected frequencies are too low:
	$$
\chi_{yates}^2 = \sum_{i} \frac{(|O_i - E_i|-0.5)^2}{E_i} \sim \chi^2 (df=(m-1)(n-1))
	$$
	\item Reject $H_0$ if $\chi^2 > \chi^2(\alpha, df=(m-1)(n-1))$
\end{itemize}
\begin{lstlisting}
prop.test(.., correct=FALSE)
\end{lstlisting}
\end{frame}


\section{Nonparametric tests}

\begin{frame}[t]
\frametitle{Nonparametric tests}
\begin{itemize}
	\item 2-groups: Mann-Whitney rank-sum test for continuous outcome in completely randomized 
		parallel design.
	\item 2-groups: Wilcoxon signed rank test for continuous outcome in crossover design.
	\item 3+-groups: Kruskal-Wallis rank-sum test for continuous outcome in completely 
		randomized parallel design. 
	\item 3+-groups: Friedman's rank sum test for continuous outcome in randomized 
		block design.
	\item 2-groups: McNemar's test for binary outcome in crossover design.
	\item 3+-groups: Cochrane's Q-test for binary outcome in crossover design.
\end{itemize}
\end{frame}


\begin{frame}[t,containsverbatim]
\frametitle{Wilcoxon-Mann-Whitney test}
\framesubtitle{Comaring two treatment group with non-normal data}
\begin{itemize}
	\item The $t$-test is usually quite robust against departures 
		from normality.
	\item However, when the departure is extreme, use 
		Mann-Whitney-Wilcoxon U-test (a.k.a Wilcoxon 
		rank-sum test)
	\item Proposed by Wilcoxon (1945) for equal sample sizes
	\item Extended to arbitrary sample sizes by Mann and Whitney 
		(1947)
	\item Virtually identical to performing ordinary parametric 
		two-sample t-test on the combined ranks.
\end{itemize}
\begin{lstlisting}
wilcox.test(...)
\end{lstlisting}
\end{frame}


\begin{frame}[t,containsverbatim]
\frametitle{Wilcoxon rank-sum test}
\framesubtitle{Nonparametric version of t-test}
When the assumption of normality and equal variances are violated, we 
may use the nonparametric test - Wilcoxon rank-sum test.

\begin{knitrout}\footnotesize
\definecolor{shadecolor}{rgb}{0.969, 0.969, 0.969}\color{fgcolor}\begin{kframe}
\begin{alltt}
\hlkwd{wilcox.test}\hlstd{(diff} \hlopt{~} \hlstd{TRT,} \hlkwc{data}\hlstd{=dbp)}
\end{alltt}


{\ttfamily\noindent\color{warningcolor}{\#\# Warning in wilcox.test.default(x = c(-9L, -15L, -21L, -14L, -11L, -15L, : cannot compute exact p-value with ties}}\begin{verbatim}
## 
## 	Wilcoxon rank sum test with continuity correction
## 
## data:  diff by TRT
## W = 0, p-value = 6.286e-08
## alternative hypothesis: true location shift is not equal to 0
\end{verbatim}
\end{kframe}
\end{knitrout}
\end{frame}

\begin{frame}[t,containsverbatim]
\frametitle{Kruskal-Wallis Test}
\framesubtitle{3+-groups of continuous outcome in randomized parallel design}
\begin{itemize}
	\item Kruskal-Wallis H test; One-way ANOVA on ranks
	\item Extension of Mann-Whitney U test
\end{itemize}

\begin{alertblock}{\center Method}
\begin{enumerate}
	\item Rank all data from all groups together
	\item Compute the statistic:
	$$
H = (N-1)\frac{\sum_{i=1}^g n_i (\bar{r}_{i*} - \bar{r})^2}{\sum_{i=1}^g \sum_{j=1}^{n_i} (r_{ij} - \bar{r})^2} \sim \chi^2(df=g-1)
	$$
	where
	\begin{itemize}
		\item $n_i$: number of observations in group $i$;
		\item $r_{ij}$: rank (among all) of observation $j$ from group $i$;
		\item $N$: total number of observations across all groups;
		\item $\bar{r}_{i*} = \frac{\sum_{j=1}^{n_i} r_{ij}}{n_i}$ is the average rank of all observations in group $i$;
		\item $\bar{r} = \frac{N+1}{2}$ is the average of all $r_{ij}$ 
	\end{itemize}
\end{enumerate}
\end{alertblock}
\begin{lstlisting}
kw.test(...)
\end{lstlisting}
\end{frame}

\begin{frame}[t,containsverbatim]
\frametitle{Friedman's Q test}
\framesubtitle{3+-groups of continuous outcome in randomized block design}
\begin{itemize}
	\item randomized block design (随机区组设计)
	\item not necessarily symmetric (不需要满足对称假设)
	\item Two-way ANOVA by ranks?
\end{itemize}
\begin{block}{\center \small Method}
\begin{itemize}
\small
	\item Given data $\{x_{ij}\}_{n \times k}$ with $n$ rows (blocks) and $k$ columns (treatments);
	\item Replace the data with new matrix $\{r_{ij}\}$, where $r_{ij}$ is the rank of $x_{ij}$ within 
		block $i$ ($r_{ij}=1,\dots,k$).
	\item Calculate the values:
	\begin{itemize}
		\item $\bar{r}_{*j} = \frac{1}{n} \sum_{i=1}^n r_{ij}$
		\item $\bar{r} = \frac{1}{nk}\sum_{i=1}^n \sum_{j=1}^k r_{ij}$
		\item $\textrm{SS}_t = n \sum_{j=1}^k (\bar{r}_{*j} - \bar{r})^2$
		\item $\textrm{SS}_e = \frac{1}{n(k-1)} \sum_{i=1}^n \sum_{j=1}^k (r_{ij} - \bar{r})^2$
	\end{itemize}
	\item Calculate the test statistic $Q = \frac{\textrm{SS}_t}{\textrm{SS}_e}$. $Q$ does not need 
		to be adjusted for tied values in the data.
	\item If $n>15$ or $k>4$, $Q \sim \chi^2_{df=k-1}$
\end{itemize}
\end{block}
\begin{lstlisting}
friedman.test(...)
\end{lstlisting}
\end{frame}


\begin{frame}[t]
\frametitle{Cochrane's Q test}
\framesubtitle{2+ correlated groups of dichotomous outcomes}
\begin{itemize}
	\item Similar to Friedman's test, but with binary outcomes.
	\item Randomized block design.
	\item $H_0: $ The treatments are all equally effective.
\end{itemize}
\begin{block}{\center \small Method}
The test statistic
$$
T = k(k-1) \frac{\sum_{j=1}^k (X_{*j} - \frac{N}{k})^2}{\sum_{i=1}^b X_{i*}(k - X_{i*})} \sim \chi^2_{df=k-1}
$$
where
\begin{itemize}
	\item $k$: number of treatments;
	\item $X_{*j}$: total for $j^{th}$ treatment;
	\item $b$: number of blocks;
	\item $X_{i*}$: row total for the $i^{th}$ block;
	\item $N$: grand total
\end{itemize}
\end{block}
\end{frame}


\section{Permutation or Resampling Approaches}

\begin{frame}[t,containsverbatim]
\frametitle{Permutation method}
\begin{enumerate}
	\item Iteratively permute the randomization (treatment assignment)
	\item Compute the statistic of interest for each permuted sample
	\item Generate the empirical distribution of the statistics
	\item Compare the original statistic with the permuted ones to determine 
		the empirical $p$-value
	\item Reject or not reject the null hypothesis
\end{enumerate}
\begin{alertblock}{\center \small Exercise}
Use \texttt{sample()} function to realize the permutation approach.
\end{alertblock}
\end{frame}


\begin{frame}[t,containsverbatim]
\frametitle{Bootstrapping method}
Boostrapping is a resampling procedure extensively used in statistics 
when any of the assumptions underlying the validity of the $t$-test 
don't hold for the data under analysis.
\begin{enumerate}
	\item Iteratively drawing samples with replacement from 
		the data.
	\item Calculating the statistic of interest for each sample
	\item Generating the empirical resampling distribution of 
		the statistic
	\item Percentile points corresponding to the Type-I error level 
		and the sided-ness of the alternative hypothesis of the 
		resampling distribution are then used in the assessment 
		of statistical significance.
\end{enumerate}
\begin{lstlisting}
bootstrap::bootstrap()
\end{lstlisting}
\end{frame}

\begin{frame}[t,containsverbatim]
\frametitle{Bootstrapping approach}
\begin{knitrout}\footnotesize
\definecolor{shadecolor}{rgb}{0.969, 0.969, 0.969}\color{fgcolor}\begin{kframe}
\begin{alltt}
\hlkwd{library}\hlstd{(bootstrap)}
\hlstd{mean.diff} \hlkwb{<-} \hlkwa{function}\hlstd{(}\hlkwc{bn}\hlstd{,} \hlkwc{dbp}\hlstd{)}
        \hlopt{-}\hlkwd{diff}\hlstd{(}\hlkwd{tapply}\hlstd{(dbp}\hlopt{$}\hlstd{diff[bn], dbp}\hlopt{$}\hlstd{TRT[bn], mean))}
\hlcom{# number of bootstrap}
\hlstd{nboot} \hlkwb{<-} \hlnum{1000}
\hlstd{boot.mean} \hlkwb{<-} \hlkwd{bootstrap}\hlstd{(}\hlnum{1}\hlopt{:}\hlkwd{dim}\hlstd{(dbp)[}\hlnum{1}\hlstd{], nboot, mean.diff, dbp)}
\hlcom{# extract the mean difference}
\hlstd{x} \hlkwb{<-} \hlstd{boot.mean}\hlopt{$}\hlstd{thetastar}
\hlkwd{quantile}\hlstd{(x,} \hlkwd{c}\hlstd{(}\hlnum{0.025}\hlstd{,} \hlnum{0.975}\hlstd{))}
\end{alltt}
\begin{verbatim}
##       2.5%      97.5% 
## -12.088384  -8.728386
\end{verbatim}
\end{kframe}
\end{knitrout}
\begin{alertblock}{\center \small Conclusion}
What conclusion can you make from the above result?
\end{alertblock}
\end{frame}

\end{CJK*}
\end{document}
