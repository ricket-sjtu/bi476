\documentclass[a4paper,11pt]{article}

\usepackage{CJK}
\usepackage{graphicx}
\usepackage{amsmath,amssymb,amsopn}
\usepackage[english]{babel}
\usepackage[latin1]{inputenc}
\usepackage{url}
\usepackage{enumerate}
\usepackage{multirow}
\usepackage{color}
\usepackage{times}
\usepackage{xcolor}
\usepackage{geometry}
\usepackage{listings}
\lstset{
language=R,
keywordstyle=\color{blue!70}\bfseries,
basicstyle=\ttfamily,
commentstyle=\ttfamily,
showspaces=false,
showtabs=false,
frame=shadowbox,
rulesepcolor=\color{red!20!green!20!blue!20},
breaklines=true
}

\geometry{paperwidth=18.4cm,paperheight=26cm}
\setlength{\parindent}{0em}
\setlength{\parskip}{1.2\baselineskip}

\title{\small{BI476: Biostatistics - Case Studies}\\
	\Large{Exercise 04 - Family of Analysis of Variance}
}
\author{Spring, 2018}
\date{}

\begin{document}
\begin{CJK*}{UTF8}{gbsn}
\maketitle


\section{Choices}
Which of the following statements is true?
\begin{enumerate}[A]
	\item The independent t-test is analogous to repeated measures ANOVA and 
		the paired-sample t-test is analogous to between-groups ANOVA.
	\item The independent t-test is analogous to between-groups ANOVA and the 
		paired-sample t-test is analogous to repeated measures ANOVA.
	\item ANOVA is analogous to t-tests, only a t-test is used when two group 
		means are compared and ANOVA is used when more than two group 
		means are compared.
	\item ANOVA and t-tests are totally different. There is no analogy
\end{enumerate}

\section{Answer the following questions}
\begin{enumerate}[(1)]
	\item What is the difference between a \textbf{between-groups} design (
		i.e. traditional ANOVA) and a \textbf{within-subject} design (i.e. 
		repeated measures ANOVA)?
	\item Suppose that you want to test the effect of alcohol on test score of 
		students. There are three conditions: no alcohol, two glasses of 
		beer, or five glasses of beer. Alcohol tolerance and time spent 
		studying should also be considered somehow. Which group design should 
		you use, between-groups design or within-groups design?
	\item Why is a repeated measures design statistically more powerful?
	\item What is the denominator degree of a $3\times 4$ ANOVA with interaction and 
		total sample size of 625? 
	\item To compute the sample size for one-way three-group ANOVA with equal variance, 
		what kinds of information do you need to pre-specify? Can you infer the 
		formula, using the similar fashion as we discuss in lecture 5.
	\item Now if the data are repeated measures, how do we change the formula to fit 
		the requirement? 
	\item How do we analyze the data from a factorial design with continous outcome?
		Write down the ANOVA table.
	\item If one of the two factors in a factorial design is ordinal, then how to 
		analyze the data?
\end{enumerate}

\section{Case study}
Use R to import the example data of the clinical trial in the lecture, and answer the 
following questions:  
\begin{enumerate}[(1)]
	\item How many different treatments did each subject receive?
	\item How many times was the diastolic blodd pressure for each subject measured?
	\item Is it appropriate to use the one-way independent ANOVA to analyze the mean 
		blood pressures across different time points? Why? How to correct it? 
		Write down the procedure and the final result. 
	\item How to assess the treatment effect of the drug $A, B$ on the blood pressure?
\end{enumerate}



\section{Case study}
A physician wanted to determine the impact of smoking on the resting 
heart rate. He randomly chose 7 individuals from each of the three 
categories: non-smokers, light-smokers (<10 cigarettes/day) and heavy 
smokers (>10 cigarettes/day) and obtained the following resting heart 
rate data (in beats/min):
\begin{table}
\begin{tabular}{lccccccc}
Nonsmoker & 56 & 53 & 53 & 65 & 70 & 58 & 51\\
Lightsmoker & 78 & 62 & 70 & 73 & 67 & 75 & 65\\
Heavysmoker & 77 & 86 & 65 & 83 & 79 & 80 & 77
\end{tabular}
\end{table}

\begin{enumerate}[(1)]
	\item Make a side-by-side boxplot showing the distribution of resting 
		heart beat for the three different groups.
	\item State the null and alternative hypotheses to test whether the mean 
		heart beat rate differs between three groups.
	\item Perform one-way ANOVA on the data. What can you conclude?
	\item If ANOVA reached a significant result, perform a post-hoc test to 
		determine which groups differ in terms of the average resting 
		heart rate. 
\end{enumerate}

\end{CJK*}
\end{document}
