\documentclass[table,10pt]{beamer}

\mode<presentation>{
%\usetheme{Goettingen}
\usetheme{Boadilla}
\usecolortheme{default}
}
\usepackage{CJK}
\usepackage{graphicx}
\usepackage{amsmath, amsopn}
\usepackage{xcolor}
\usepackage[english]{babel}
\usepackage[T1]{fontenc}
\usepackage[latin1]{inputenc}
\usepackage{enumerate}
\usepackage{multirow}
\usepackage{url}
\ifx\hypersetup\undefined
	\AtbBeginDocument{%
		\hypersetup{unicode=true,pdfusetitle,
bookmarks=true,bookmarksnumbered=false,bookmarksopen=false,
breaklinks=false,pdfborder={0 0 0},pdfborderstyle={},backref=false,colorlinks=false}
	}
\else
	\hypersetup{unicode=true,pdfusetitle,
bookmarks=true,bookmarksnumbered=false,bookmarksopen=false,
breaklinks=false,pdfborder={0 0 0},pdfborderstyle={},backref=false,colorlinks=false}
\fi
\usepackage{breakurl}
\usepackage{color}
\usepackage{times}
\usepackage{xcolor}
\usepackage{listings}
\lstset{
language=R,
keywordstyle=\color{blue!70}\bfseries,
basicstyle=\ttfamily,
commentstyle=\ttfamily,
showspaces=false,
showtabs=false,
frame=shadowbox,
rulesepcolor=\color{red!20!green!20!blue!20},
breaklines=true}



\setlength{\parskip}{.5em}

\title[BI476]{BI476: Biostatistics - Case Studies}
\subtitle[anova]{Lec05: Power and Sample Size Calculation}
\author[Maoying Wu]{Maoying,Wu ({\small ricket.woo@gmail.com})}
\institute[CBB] % (optional, but mostly needed)
{
  \inst{}
  Dept. of Bioinformatics \& Biostatistics\\
  Shanghai Jiao Tong University
}
\date{Spring, 2018}

\AtBeginSection[]
{
  \begin{frame}<beamer>{Next Section ...}
    \tableofcontents[currentsection]
  \end{frame}
}

\begin{document}
\begin{CJK*}{UTF8}{gbsn}

\begin{frame}
\titlepage
\end{frame}

\begin{frame}[t]
\frametitle{Example}
\begin{table}
\caption{A multi-center two-arm randomized phase III trial}
\small
\begin{tabular}{p{0.30\textwidth}p{0.60\textwidth}}
Study design & A multi-center two-arm randomized phase III trial to compare the combination of 
		gemcitabine (吉西它宾) and docetaxel (多西它赛) versus gemcitabine along\\
Patients & Advanced or metastatic unresectable soft tissue sarcoma (软组织肉瘤)\\
Primary outcome & progression-free survival\\
Secondary outcome & overall survival
\end{tabular}
\end{table}
\end{frame}

\begin{frame}[t]
\frametitle{Why sample size?}
\begin{itemize}
	\item The aim of a clinical trial is to judge the efficacy and/or 
		safety of a new therapy or drug.
	\item In the planning phase of the study, the calculation of the 
		necessary sample size is crucial in order to obtain a 
		meaningful result.
	\item The study design, the expected treatment \alert{effect} in outcome 
		and its \alert{variability}, \alert{power} and \alert{significance level} 
		are factors which determine the sample size.
	\item It is often difficult to fix these parameters prior to the start of the 
		study, but related papers from the literature can be helpful sources for 
		the unknown quantities.
	\item For scientific as well as ethical reasons it is necessary to calculate 
		the sample size in advance in order to be able to answer the study 
		question.
\end{itemize}
\end{frame}


\begin{frame}[t]
\frametitle{Calculation of sample size}
\begin{table}
\caption{Type I error, Type II error, p-value and power}
\begin{tabular}{lll}
& $H_0$ is true & $H_1$ is true\\
\hline
Reject $H_0$ & $\alpha$(Type I error) & $1-\beta$(power)\\
Not Reject $H_0$ & & $\beta$(Type II error)\\
\hline 
\end{tabular}
\end{table}
\begin{table}
\begin{tabular}{lcl}
\textbf{Power} & = & The probability of rejecting $H_0$ if $H_1$ is true.\\
\textbf{Power} & = & $f$($\alpha$, variation, clinically significant level, \alert{Sample size})\\
\textbf{Sample size} & = & $f$(\alert{Power}, $\alpha$, variation, clinically significant level)\\
$p$-value & = & the probability of observing this difference if $H_0$ were true.
\end{tabular}
\end{table}
\end{frame}


\begin{frame}[t]
\frametitle{An example}
\framesubtitle{Manuscript writeup}
\alert{\Large Cognitive therapy for the prevention of suiside attempts: a randomized controlled trial.}\\
{\small Brown GK et al. JAMA 2005; 294:463-570.}

To test the primary hypothesis that the mean time to the next suicide 
attempt during the follow-up period is different between treatment 
groups, a priori power calculations were based on the results of a 
previous randomized controlled trial with a similar protocol. The current sample size ($N = 120$) provided at least $80\%$ power to 
detect a hazard ratio of $0.44$ in terms of time to next suicide 
attempt between treatment groups using an assumed repeat attempt rate 
of $25.8\%$ during the follow-up period and a two-sided $\alpha$ level of 0.05.
\end{frame}

\begin{frame}[t]
\frametitle{1. Comparing means for continuous outcomes}
$$
Y_{ik} \sim N(\mu_k, \sigma^2); k=1,2; i=1,\dots,n_k
$$
\begin{itemize}
	\item Testing for equality of two independent means
	\item Superiority trial of two independent means
	\item Non-inferiority trial of two independent means
	\item Equivalance trial of two independent means
\end{itemize}
\end{frame}

\begin{frame}
\frametitle{1.1 Testing for equality of two means}
$$
H_0: \theta = \mu_1 - \mu_2 = 0 \mbox{ versus } H_1: \theta \neq 0
$$

\begin{alertblock}{\center Procedure}
\begin{enumerate}
	\item The sample mean: $\bar{Y}_k = \frac{1}{n_k} \sum_{i=1}^{n_k} Y_{ik}, k=1,2$
	\item The pooled-sample variance: $s_n^2 = \frac{1}{n_1 + n_2 - 2} \sum_{k=1}^2 \sum_{i=1}^{n_k} (Y_{ik} - \bar{Y}_k)^2$
	\item The two-sample $t$-test statistic: $T_n = \frac{\bar{Y}_1 - \bar{Y}_2}{s_n \sqrt{1/n_1 + 1/n_2}}$
	\item Under $H_0$: $T_n \sim t_{\nu}, \nu = n_1 + n_2 - 2$;
	\item Under $H_1$: $T_n \sim t_{\nu}(c), c=\frac{\theta}{\sigma \sqrt{1/n_1 + 1/n_2}}$
	\item Reject $H_0$ if $|T_n| \ge t_{\nu, \alpha/2}$
\end{enumerate}
\end{alertblock}
\end{frame}


\begin{frame}[t]
\frametitle{1.1 Testing for equality of two means}
Under $H_1$, the power of the two-sample $t$-test is given by
$$
1 - \beta = 1 - \mathcal{T}(t_{\nu, \alpha/2}, c) + \mathcal{T}(-t_{\nu,\alpha/2}, c)
$$
where $\mathcal{T}(\dot,c)$ denotes the cumulative distribution function of the noncentral $t_{\nu}(c)$ distribution.

We specify the treatment difference to be detected as $\theta$ and define
$$
Z = \frac{\bar{Y}_1 - \bar{Y}_2 - \theta}{\sigma \sqrt{1/n_1 + 1/n_2}} \sim N(0, 1)
$$

Now the power is given by
$$
\begin{array}{lcl}
1 - \beta &=& Pr \left( |\frac{\bar{Y}_1 - \bar{Y}_2}{\sigma \sqrt{1/n_1 + 1/n_2}}| \ge z_{\alpha/2} \big| H_1\right)\\
&=& Pr \left( Z \ge z_{\alpha/2} - \frac{\theta}{\sigma\sqrt{1/n_1 + 1/n_2}} \big| H_1\right)\\
& & + Pr \left( Z \le -z_{\alpha/2} - \frac{\theta}{\sigma\sqrt{1/n_1 + 1/n_2}} \big| H_1\right)
\end{array}
$$ 

\end{frame}



\begin{frame}[t]
\frametitle{1.1 Testing for equality of two means}
When $\theta > 0$, we can ignore the second term
$$
\beta \approx Pr\left( Z \le z_{\alpha/2} - \frac{\theta}{\sigma \sqrt{1/n_1 + 1/n_2}} \big| H_1\right)
$$
Similarly when $\theta < 0$, we can ignore the first term
$$
\beta \approx Pr\left(Z \le z_{\alpha/2} + \frac{\theta}{\sigma \sqrt{1/n_1 + 1/n_2}} \big| H_1 \right)
$$
That is
$$
\beta \approx \Phi\left( z_{\alpha/2} - \frac{|\theta|}{\sigma \sqrt{1/n_1 + 1/n_2}} \right)
$$
Therefore, sample size can be obtained by solving
$$
-z_{\beta} = z_{\alpha/2} - \frac{|\theta|}{\sigma \sqrt{1/n_1 + 1/n_2}} \Rightarrow \frac{|\theta|}{\sigma \sqrt{1/n_1 + 1/n_2}} = z_{\alpha/2} + z_{\beta}
$$
If $n_1 = n_2 = n$,
$$
n = \frac{2\sigma^2(z_{\alpha/2}+z_{\beta})^2}{\theta^2} 
$$
\end{frame}


\begin{frame}[t]
\frametitle{1.1 Exercise: Unbalanced allocation}
If the patience allocation ratio between arm 1 and arm 2 is
$$
r = n_1/n_2
$$
\begin{itemize}
	\item Calculate the required total sample size.
	\item Is it larger or smaller than the balanced allocation?
	\item If we set $\theta=1, \sigma^2 = 4, \alpha=0.05$, calculate the required sample size 
		for balanced allocation and unbalanced allocation with $r=2$, resepectively.
	\item Draw a figure to illustrate how the powers change with changing allocation by fixing 
		the total sample size to be 200 and $\alpha=0.05$. 
\end{itemize}
\end{frame}


\begin{frame}[t]
\frametitle{1.2 Superiority Trial}
A superiority trial can be formulated as a one-sided test:
$$
H_0: \theta \le 0 \mbox{ versus } H_1: \theta > 0
$$
where $\theta$ is the preset clinically meaningful difference and we define
$$
Z = \frac{\bar{Y}_1 - \bar{Y}_2 - \theta}{\sigma \sqrt{1/n_1 + 1/n_2}}
$$
The power is given by
$$
\begin{array}{lcl}
1 - \beta &=& Pr \left( \frac{\bar{Y}_1 - \bar{Y}_2}{\sigma\sqrt{1/n_1 + 1/n_2}} \ge z_{\alpha} \big| H_1\right)\\
&=& Pr \left( Z \ge z_{\alpha} - \frac{\theta}{\sigma \sqrt{1/n_1 + 1/n_2}} \big| H_1 \right)
\end{array}
$$
which leads to
$$
\frac{\theta}{\sigma \sqrt{1/n_1 + 1/n_2}} = z_{\alpha} + z_{\beta}
$$
If $n_1 = n_2 = n$, then
$$
n = \frac{2\sigma^2(z_{\alpha} + z_{\beta})^2}{\theta^2}
$$
\end{frame}

\begin{frame}[t]
\frametitle{1.3 Non-inferiority Trial}
A non-inferiority trial can be formulated as a one-sided test:
$$
H_0: \theta \le -\delta \mbox{ versus } H_1: \theta > -\delta
$$
where $\delta > 0$ is the noninferiority margin.

$H_0$ is rejected when
$$
\frac{\bar{Y}_1 - \bar{Y}_2 + \delta}{\sigma \sqrt{1/n_1 + 1/n_2}} \ge z_{\alpha}
$$
and define
$$
Z = \frac{\bar{Y}_1 - \bar{Y}_2 - \theta}{\sigma \sqrt{1/n_1 + 1/n_2}}
$$
and the power is given by
$$
\begin{array}{lcl}
1-\beta &=& Pr\left( \frac{\bar{Y}_1 - \bar{Y}_2 + \delta}{\sigma \sqrt{1/n_1 + 1/n_2}} \ge z_{\alpha} \big| H_1 \right)\\
&=& Pr \left( Z \ge z_{\alpha} - \frac{\delta + \theta}{\sigma \sqrt{1/n_1 + 1/n_2}} \big| H_1 \right)
\end{array}
$$
leading to
$$
\frac{\theta + \delta}{\sigma \sqrt{1/n_1 + 1/n_2}} = z_{\alpha} + z_{\beta} \Rightarrow n = \frac{2\sigma^2(z_{\alpha} + z_{\beta})^2}{(\theta + \delta)^2}
$$
\end{frame}


\begin{frame}[t]
\frametitle{1.4 Equivalence Trial}
An equivalence trial can be formulated as
$$
H_0: | \theta | \ge \delta \mbox{ versus } H_1: |\theta| < \delta
$$
The rejection region can be written as
$$
z_{\alpha} - \frac{\delta}{\sigma \sqrt{1/n_1+1/n_2}} \le \frac{\bar{Y}_1 - \bar{Y}_2}{\sigma \sqrt{1/n_1+1/n_2}} \le -z_{\alpha} + \frac{\delta}{\sigma \sqrt{1/n_1+1/n_2}}
$$
Let
$$
Z = \frac{\bar{Y}_1 - \bar{Y}_2 - \theta}{\sigma \sqrt{1/n_1+1/n_2}} \sim N(0,1)
$$
And the power is given by
$$
\begin{array}{lcl}
1-\beta &=& Pr \left( z_{\alpha} - \frac{\delta + \theta}{\sigma \sqrt{1/n_1+1/n_2}} \le Z \le - z_{\alpha} + \frac{\delta - \theta}{\sigma \sqrt{1/n_1+1/n_2}} \big| H_1 \right)\\
&=& \Phi\left( -z_{\alpha} + \frac{\delta - \theta}{\theta \sqrt{1/n_1 + 1/n_2}}\right) - \Phi \left( z_{\alpha} - \frac{\delta+\theta}{\sigma \sqrt{1/n_1 + 1/n_2}}\right)  
\end{array}
$$
\end{frame}

\begin{frame}[t]
\frametitle{1.4 Equivalence Trial}
In a conservative derivation with no power inflation, we have
$$
1-\beta \approx 2 \Phi\left( -z_{\alpha} + \frac{\delta - |\theta|}{\sigma \sqrt{1/n_1 + 1/n_2}} \right) - 1
$$
which leads to
$$
\frac{\delta - |\theta|}{\sigma \sqrt{1/n_1 + 1/n_2}} = z_{\alpha} + z_{\beta/2}
$$
If $n_1 = n_2 = n$, then
$$
n = \frac{2\sigma^2(z_{\alpha} + z_{\beta/2})^2}{(\delta - |\theta|)^2}
$$

Note: $\theta$ is often set to zero in practice.
\end{frame}


\begin{frame}[t]
\frametitle{1.4 Exercise: Bioequivalence trial sample size}
In a clinical trial with cardiovascular disease, both the novel and 
standard therapies target lowering the blood pressure. The study is to 
establish equivalence of the two treatments in terms of therapeutic 
effects with an equivalence margin $\delta=0.2$. The variance of the 
medical measurements is estiamted to be 1.0 from previous study.

In an equivalence trial often uses the $90\%$ rather than $95\%$ 
confidence interval.

Then to achieve a power of $90\%$, how many total sample size do we need?
\end{frame}


\begin{frame}[t]
\frametitle{2. Comparing proportions for binary outcomes}
Let $Y_{ik}$ be the binary outcome for subject $i=1,\dots,n_k$ in arm $k=1,2$:
$$
Y_{ik} = \begin{cases}
1, & \mbox{ with probability } p_k,\\
0, & \mbox{ with probability } 1-p_k,
\end{cases}
$$
and the sum in each arm $k$ is
$$
\sum_{i=1}^{n_k} Y_{ik} \sim Bin(n_k, p_k)
$$
The sample proportion for arm $k$ is
$$
\begin{array}{lcl}
\bar{Y}_k &=& \frac{1}{n_k} \sum_{i=1}^{n_k} Y_{ik}\\
E(\bar{Y}_k) &=& p_k\\
Var(\bar{Y}_k) &=& p_k(1-p_k)/n_k
\end{array}
$$
The difference
$$
\theta = p_1 - p_2
$$
\end{frame}


\begin{frame}[t]
\frametitle{2.1 Test for the equation of two proportions}
$$
H_0: \theta = 0 \mbox{ versus } H_1: \theta \neq 0
$$
Under $H_0$, the test statistic
$$
T_n = \frac{\bar{Y}_1 - \bar{Y}_2}{\sqrt{\bar{Y}(1-\bar{Y})(1/n_1 + 1/n_2)}} \sim N(0, 1),
$$
where $\bar{Y}$ is the pooled-sample mean:
$$
\bar{Y} = \frac{n_1 \bar{Y}_1 + n_2 \bar{Y}_2}{n_1 + n_2}
$$

While under the alternative hypothesis,
$$
T_n \big| H_1 \sim N \left(\frac{\theta}{\sqrt{\bar{p}(1-\bar{p})(1/n_1 + 1/n_2)}}, \frac{p_1(1-p_1)/n_1 + p_2(1-p_2)/n_2}{\bar{p}(1-\bar{p})(1/n_1 + 1/n_2)}\right)
$$
\end{frame}


\begin{frame}[t]
\frametitle{2.1 Test for the equation of two proportions}
Define the standard normal random variable
$$
Z = \frac{\bar{Y}_1 - \bar{Y}_2 - \theta}{\sqrt{p_1(1-p_1)/n_1 + p_2(1-p_2)/n_2}}
$$
where $\theta$ is the treatment difference to be detected.

The power is therefore
$$
\begin{array}{lcl}
1-\beta &=& Pr \left( \big| \frac{\bar{Y}_1 - \bar{Y}_2}{\sqrt{\bar{p} (1-\bar{p})(1/n_1+1/n_2)}}\big| \ge z_{\alpha/2} \big| H_1\right)\\
&=& Pr \left( Z \ge \frac{z_{\alpha/2} \sqrt{\bar{p}(1-\bar{p})(1/n_1 + 1/n_2)} - \theta}{\sqrt{p_1(1-p_1)/n_1 + p_2(1-p_2)/n_2}} \big| H_1 \right)\\
 & & + Pr \left( Z \le \frac{-z_{\alpha/2} \sqrt{\bar{p}(1-\bar{p})(1/n_1 + 1/n_2)} - \theta}{\sqrt{p_1(1-p_1)/n_1 + p_2(1-p_2)/n_2}} \big| H_1 \right)\\
\end{array}
$$
\end{frame}

\begin{frame}[t]
\frametitle{2.1 Test for the equality of two proportions}
Therefore
$$
\beta \approx \Phi \left( \frac{z_{\alpha/2}\sqrt{\bar{p}(1-\bar{p})(1/n_1 + 1/n_2)} - |\theta|}{\sqrt{p_1(1-p_1)/n_1 + p_2(1-p_2)/n_2}} \right),
$$
and the sample size can be obtained by solving
$$
|\theta| = z_{\alpha/2} \sqrt{\bar{p}(1-\bar{p})(1/n_1+1/n_2)} + z_{\beta} \sqrt{p_1(1-p_1)/n_1 + p_2(1-p_2)/n_2}
$$
If $n_1 = n_2 = n$, then the sample size
$$
n = \frac{\left( z_{\alpha/2}\sqrt{2\bar{p}(1-\bar{p})} + z_{\beta} \sqrt{p_1(1-p_1) + p_2(1-p_2)}\right)^2}{\theta^2}
$$
\end{frame}


\begin{frame}[t]
\frametitle{2.2 Sample size with unpooled variance}
Under the two-sided hypothesis,
$$
T_n = \frac{\bar{Y}_1 - \bar{Y}_2}{\sqrt{\bar{Y}_1(1-\bar{Y}_1)/n_1 + \bar{Y}_2(1-\bar{Y}_2)/n_2}}
$$
which can be approximated by
$$
T_n \approx \frac{\bar{Y}_1 - \bar{Y}_2}{\sqrt{p_1(1-p_1)/n_1 + p_2(1-p_2)/n_2}}
$$
Under $H_0$,
$$
T_n \big| H_0 \sim N(0, 1)
$$
and under $H_1$,
$$
T_n \big| H_1 \sim N \left( \frac{\theta}{\sqrt{p_1(1-p_1)/n_1 + p_2(1-p_2)/n_2}},1 \right)
$$
\end{frame}


\begin{frame}[t]
\frametitle{2.2 Sample size with unpooled variance}
The power can be calculated as
$$
\begin{array}{lcl}
1-\beta &=& Pr \left( \big| \frac{\bar{Y}_1 - \bar{Y}_2}{\sqrt{p_1 (1-p_1)/n_1+ p_2(1-p_2)/n_2)}} \big| \ge z_{\alpha/2} \big| H_1 \right)\\
&=& Pr \left( Z \ge z_{\alpha/2} - \frac{\theta}{\sqrt{p_1(1-p_1)/n_1 + p_2(1-p_2)/n_2}} \big| H_1 \right)\\
 & & + Pr \left( Z \le -z_{\alpha/2} - \frac{\theta}{\sqrt{p_1(1-p_1)/n_1 + p_2(1-p_2)/n_2}} \big| H_1 \right)
\end{array}
$$
Therefore,
$$
\beta \approx \Phi \left( z_{\alpha/2} - \frac{|\theta|}{\sqrt{p_1(1-p_1)/n_1 + p_2(1-p_2)/n_2}}\right)
$$
and the sample size can be obtained by solving
$$
\frac{|\theta|}{\sqrt{p_1(1-p_1)/n_1 + p_2(1-p_2)/n_2}} = z_{\alpha/2} + z_{\beta}
$$
If $n_1 = n_2 = n$, then
$$
n = \frac{(z_{\alpha/2}+z_{\beta})^2}{\theta^2}\left[ p_1(1-p_1) + p_2(1-p_2)\right]
$$
\end{frame}

\begin{frame}[t]
\frametitle{2.2 Exercise: Sample size estimation}
Suppose that the standard treatment has a response rate of $30\%$ for metastatic breast cancer patients, and the new treatment is expected to 
have an improvemment of $10\%$.

We set type I error rate of $\alpha=0.05$ and power $90\%$. How many 
patients do we need? Use both the pooled and unpooled variance 
estimators.
\end{frame}


\begin{frame}[t]
\frametitle{2.3 Superiority Trial}
A superiority trial can be formulated as a one-sided hypothesis:
$$
H_0: \theta \le 0 \mbox{ versus } H_1: \theta > 0.
$$
Here the power is given by
$$
1-\beta = Pr \left( Z \ge \frac{z_{\alpha} \sqrt{\bar{p}(1-\bar{p})(1/n_1 + 1/n_2)} - \theta}{\sqrt{p_1(1-p_1)/n_1 + p_2(1-p_2)/n_2}} \big| H_1\right)
$$
where $Z$ is the standard normal variable defined as
$$
Z = \frac{\bar{Y}_1 - \bar{Y}_2 - \theta}{\sqrt{p_1(1-p_1)/n_1 + p_2(1-p_2)/n_2}}
$$
Therefore, the sample size can be obtained by solving
$$
\theta = z_{\alpha} \sqrt{\bar{p}(1-\bar{p})(1/n_1+1/n_2)} + z_{\beta} \sqrt{p_1(1-p_1)/n_1 + p_2(1-p_2)/n_2}
$$
If $n_1 = n_2 = n$, then
$$
n = \frac{\left[ z_{\alpha} \sqrt{2\bar{p}(1-\bar{p})} + z_{\beta} \sqrt{p_1(1-p_1) + p_2(1-p_2)}\right]^2}{\theta^2}
$$
\end{frame}


\begin{frame}[t]
\frametitle{2.3 Exercise: Unbalanced allocation for superiority trial for binary outcome}
\begin{enumerate}[(1)]
	\item If we do NOT pool two samples for estimating the variance, and the allocation 
		ratio is $r$. Write down the required total sample size here for a set of 
		given parameters $\{\alpha, \beta, \theta, p_1, p_2, r\}$.
	\item The response rate of a standard chemotherapy for prostate cancer is $20\%$, and 
		the experimental drug would double the response rate. In a one-sided test, we 
		specify the type I error rate $\alpha=0.025$ and power $90\%$. How many samples 
		do we need in such as superirority trial? Compute based on both balanced and 
		unbalanced allocation of $r=3$ (i.e., $n_1 = 3n_2$).
\end{enumerate}
\end{frame}


\begin{frame}[t]
\frametitle{2.4 Noninferiority Trial}
A noninferiority trial can be formulated as a one-sided hypothesis:
$$
H_0: \theta \le -\delta \mbox{ versus } H_1: \theta > \delta.
$$
$H_0$ is rejected at a significance level of $\alpha$, if
$$
\frac{\bar{Y}_1 - \bar{Y}_2 + \delta}{\sqrt{\bar{p}(1-\bar{p})(1/n_1+1/n_2)}} \ge z_{\alpha}.
$$
Here the power is given by
$$
1-\beta = Pr \left( Z \ge \frac{z_{\alpha} \sqrt{\bar{p}(1-\bar{p})(1/n_1 + 1/n_2)} - (\theta+\delta)}{\sqrt{p_1(1-p_1)/n_1 + p_2(1-p_2)/n_2}} \big| H_1\right)
$$
where $Z$ is the standard normal variable defined as
$$
Z = \frac{\bar{Y}_1 - \bar{Y}_2 - \theta}{\sqrt{p_1(1-p_1)/n_1 + p_2(1-p_2)/n_2}}
$$
Therefore, the sample size can be obtained by solving
$$
\theta + \delta = z_{\alpha} \sqrt{\bar{p}(1-\bar{p})(1/n_1+1/n_2)} + z_{\beta} \sqrt{p_1(1-p_1)/n_1 + p_2(1-p_2)/n_2}
$$
If $n_1 = n_2 = n$, then $
n = \frac{\left[ z_{\alpha} \sqrt{2\bar{p}(1-\bar{p})} + z_{\beta} \sqrt{p_1(1-p_1) + p_2(1-p_2)}\right]^2}{(\theta + \delta)^2}
$

In practice, we often set $\theta = 0$.
\end{frame}


\begin{frame}[t]
\frametitle{2.4 Exercise: Noninferiority trial}
We are interested in testing whether a new treatment is noninferior to 
the standard care, while less toxic and easier to administer. Suppose 
that the estimated difference of the response rates between the active 
control and placebo is $20\%$, with a $95\%$ confidence interval of 
$[0.16, 0.24]$.

We may set $\delta$ as a half of the minimal estimated difference 
between the active control and placebo (the lower bound of $95\%CI$), 
that is, $\delta=0.08$.

For a one-sided test with $\alpha=0.025$ and power=$80\%$. How to 
estimate the required sample size?
\end{frame}


\begin{frame}[t]
\frametitle{2.5 Equivalence Trial}
An equivalence trial can be formulated as one-sided trial:
$$
H_0: |\theta| \ge \delta \mbox{ versus } H1: |\theta| < \delta
$$
The null hypothesis is rejected at a significance level of $\alpha$, if
$$
z_{\alpha} - \frac{\delta}{\sqrt{\bar{p}(1-\bar{p})(1/n_1+1/n_2)}} \le T_n \le -z_{\alpha} + \frac{\delta}{\sqrt{\bar{p}(1-\bar{p})(1/n_1+1/n_2)}}
$$
Under $H_1$, the power is given by:
$$
1-\beta \approx 2\Phi\left( \frac{-z_{\alpha}\sqrt{\bar{p}(1-\bar{p})(1/n_1 + 1/n_2)} + \delta - |\theta|}{\sqrt{p_1(1-p_1)/n_1+p2(1-p_2)/n_2}}\right) - 1
$$
Therefore, the sample size can be obtained by solving
$$
\delta - |\theta| = z_{\alpha} \sqrt{\bar{p}(1-\bar{p})(1/n_1+1/n_2)} + z_{\beta/1} \sqrt{p_1(1-p_1)/n_1 + p_2(1-p_2)/n_2}
$$
If $n_1 = n_2 = n$, then
$$
n = \frac{\left[ z_{\alpha \sqrt{\bar{p}(1-\bar{p})}} + z_{\beta/2} \sqrt{p_1(1-p_1)+p_2(1-p_2)}\right]^2}{(\delta - |\theta|)^2}
$$
\end{frame}


\begin{frame}[t]
\frametitle{Power and sample size analysis in R: \texttt{pwr}}
\begin{table}
\begin{tabular}{ll}
\textbf{function} & \textbf{power calculation for}\\
\hline
\texttt{pwr.2p.test} & two proportions (equal $n$)\\
\texttt{pwr.2p2n.test} & two proportions (unequal $n$)\\
\texttt{pwr.anova.test} & balanced one-way ANOVA\\
\texttt{pwr.chisq.test} & chi-squared test\\
\texttt{pwr.f2.test} & general linear model\\
\texttt{pwr.p.test} & proportion (one-sample)\\
\texttt{pwr.r.test} & correlation test\\
\texttt{pwr.t.test} & t-tests (one-sample, two-sample, paired)\\
\texttt{pwr.t2n.test} & t-tests (two-samples with unequal $n$)\\
\hline
\end{tabular}
\end{table}
\end{frame}


\begin{frame}[t]
\frametitle{Exercise: Fill in the sample size table below}
Use $\alpha=0.025$ and two-sided test for calculations.
\begin{table}
\begin{tabular}{llcc}
$p(y=1|exposed)$ & $p(y=1|nonexposed)$ & Power & Sample size\\
\hline
0.1 & 0.5 & 0.90 & ?\\
0.3 & 0.5 & 0.90 & ?\\
0.45 & 0.5 & 0.90 & ?\\
0.2 & 0.8 & 0.80 & ?\\
0.4 & 0.8 & 0.80 & ?\\
0.6 & 0.8 & 0.80 & ?\\
0.2 & 0.8 & 0.70 & ?\\
0.4 & 0.8 & 0.70 & ?\\
0.6 & 0.8 & 0.70 & ?\\
\hline
\end{tabular}
\end{table}
\end{frame}

\begin{frame}[t]
\frametitle{After-class Exercises}
\begin{enumerate}[(1)]
	\item In a superiority trial, suppose that the response rate of 
		the standard treatment is $25\%$, and we expect that of 
		the experimental treatment to be $40\%$. We set type I 
		error rate $\alpha=0.025$ and a power of $90\%$. What is
		the total sample size if th allocation ratio between the
		experimental and standard arms is 2:1? Compare the 
		sample size estimation based on the two different 
		formulae (pooled or unpooled). How about the sample 
		size using an allocation ratio of 1:1?
	\item In a noninferiority trial, suppose that the noninferiority
		margin $\delta=0.05$, and we take the response rates of 
		both the standard and experimental treatments to be 
		$30\%$. For a type I error rate of $\alpha=0.05$ and a 
		power of $90\%$, what is the total sample size of the 
		trial with an allocation ratio of 3:1?  
\end{enumerate}
\end{frame}


\begin{frame}[t]
\frametitle{Power and sample size: Summary}
\begin{enumerate}[<+->]
	\item Carefully consider and state the null hypothesis to be tested.
	\item Select an appropriate statistical tests:
	\begin{itemize}
		\item e.g., paired t-test
	\end{itemize}
	\item Choose a clinically significant effect size.
	\item Provide an estimate of the variability.
	\begin{itemize}
		\item according to previous studies or literature.
	\end{itemize}
 	\item Select the type I error ($\alpha$) and required power ($1-\beta$) you 
		are willing to accept.
	\item Calcualte the sample size.
	\item Adjust sample size estimate for dropout if necessary.
\end{enumerate}
\end{frame}

\end{CJK*}
\end{document}
